The ranges library supports function composition using the | symbol.

\filename{Ranges work directly on the container}

\begin{cpp}
// rangesComposition.cpp
...
#include <ranges>
std::map<std::string, int> freqWord{ {"witch", 25}, {"wizard", 33}, {"tale", 45},
									 {"dog", 4}, {"cat", 34}, {"fish", 23} };

std::cout << "All words: "; // (1)
for (const auto& name : std::views::keys(freqWord)) { std::cout << name << " "; } \

std::cout << "All words reverse: "; // (2)
for (const auto& name : std::views::keys(freqWord) | std::views::reverse) {
	std::cout << name << " ";
}

std::cout << "The first 4 words: "; // (3)
for (const auto& name : std::views::keys(freqWord) | std::views::take(4)) {
	std::cout << name << " ";
}

std::cout << "All words starting with w: "; // (4)
auto firstw = [](const std::string& name){ return name[0] == 'w'; };
for (const auto& name : std::views::keys(freqWord) | std::views::filter(firstw)) {
	std::cout << name << " ";
}
\end{cpp}

In this case, I’m only interested in the keys. I display all of them (1), all of them reversed (2), the first four (3), and the keys starting with the letter ‘w’ (4).

The pipe symbol | is syntactic sugar for function composition. Instead of C(R), you can write R | C. Consequentially, the next three lines are equivalent.

\begin{cpp}
auto rev1 = std::views::reverse(std::views::keys(freqWord));
auto rev2 = std::views::keys(freqWord) | std::views::reverse;
auto rev3 = freqWord | std::views::keys | std::views::reverse;
\end{cpp}

Finally, here is the output of the program:

\myGraphic{0.6}{content/chapter11/images/3.jpg}{}

















