C++ and, therefore, the standard library has a long history. C++ started in the 1980s of the last millennium and ended now in 2023. Anyone who knows about software development knows how fast our domain evolves. So 40 years is a very long period. You may not be so astonished that the first components of C++, like I/O streams, were designed with a different mindset than the modern Standard Template Library (STL). C++ started as an object-oriented language, incorporated generic programming with the STL, and has now adopted many functional programming ideas. This evolution of software development in the last 40 years, which you can observe in the C++ standard library, is also an evolution in how software problems are solved.

\myGraphic{1.0}{content/chapter1/images/1.jpg}{C++ timeline}

The first C++98 standard library from 1998 had three components. Those were the previously mentioned I/O streams, mainly for file handling, the string library, and the Standard Template Library.

The Standard Template Library facilitates the transparent application of algorithms on containers. The history continues in 2005 with Technical Report 1 (TR1). The extension to the C++ library ISO/IEC TR 19768 was not an official standard, but almost all components became part of C++11. These were, for example, the libraries for regular expressions, smart pointers, hash tables, random numbers, and time, based on the boost libraries (http://www.boost.org/).

In addition to the standardization of TR1, C++11 got one new component: the multithreading library.

C++14 was only a minor update to the C++11 standard. Therefore, C++14 added only a few improvements to existing libraries for smart pointers, tuples, type traits, and multithreading.

C++17 includes libraries for the file system and the two new data types std::any and std::optional.

C++20 has four outstanding features: concepts, ranges, coroutines, and modules. Besides the big four, there are more pearls in C++20: the three-way comparison operator, the formatting library, and the concurrency-related data types semaphores, latches, and barriers.

C++23 improved the big four of C++20: extended ranges functionality, the coroutine generator std::generator, and a modularized C++ standard library.












