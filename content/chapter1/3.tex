要在程序中使用库,必须执行三个步骤。首先,必须使用\#include语句包含头文件,以便编译器知道库的名称。或者,可以使用import std;在c++ 23中导入标准库。由于C++标准库名称位于命名空间std中,因此可以在第二步中使用限定的名称,或者必须在全局命名空间中导入。第三步也是最后一步是指定链接器获取可执行文件所需的库。第三步通常是可选的。下面几行解释了这三个步骤。

\mySamllsection{包含头文件}

预处理器在\#include语句之后包含该文件。在大多数情况下,这是一个头文件。头文件放在尖括号中:

\begin{cpp}
#include <iostream>
#include <vector>
\end{cpp}

\begin{myWarning}{指定所有必要的头文件}
编译器可以自由地向头文件中添加额外的头文件。因此,程序可能拥有所有必需的头文件,尽管没有指定它们。不建议依赖此功能。所有需要的头文件都应该明确指定。否则,编译器升级或代码移植可能会引发编译错误。
\end{myWarning}

\mySamllsection{导入标准库}

C++23中,可以使用import语句导入整个标准库:

\begin{cpp}
import std;
\end{cpp}

import std;语句从c++头文件和C包装头文件(如std::printf from <cstdio>)中导入命名空间std中的所有内容。此外,还从<new>报头中导入::operator new或::operator delete。

若还想从C包装器头文件<stdio.h>中导入全局命名空间对应项,例如::printf,请使用import std.compat。

\begin{myWarning}{比起包含头文件,更推荐导入模块}

模块与头文件相比有许多优点。这些优点不仅适用于模块化的标准库,也适用于用户定义的模块。

模块只导入一次,而且这个过程实际上无开销。导入模块的顺序没有区别,并且模块的重复名称是非常不可能的。模块能够表达代码的逻辑结构,可以显式指定应该导出或不导出的名称。此外,可以将几个模块捆绑成一个更广泛的模块,并将它们作为一个逻辑包提供给用户。

\end{myWarning}

\begin{myWarning}{并行使用头文件和模块}

在以下代码片段中,我使用了两种语法形式:头文件和模块。你应该只使用其中一种。

\end{myWarning}

\mySamllsection{使用命名空间}

若使用限定名,则应精确地按照定义使用它们。对于每个命名空间,必须使用范围解析操作符::。C++标准库的更多库使用嵌套命名空间。

\begin{cpp}
#include <iostream> // either
#include <chrono> // either
...
#import std; // or
...
std::cout << "Hello world:" << '\n';
auto timeNow= std::chrono::system_clock::now();
\end{cpp}

\mySamllsection{使用非限定名称}

C++中,可以通过using声明和using指令来使用名称。

\mySamllsection{使用声明}

using声明将一个名称添加到应用using声明的可见性范围中:

\begin{cpp}
#include <iostream> // either
#include <chrono> // either
...
#import std; // or
...
using std::cout;
using std::endl;
using std::chrono::system_clock;
...
cout << "Hello world:" << endl; // unqualified name
auto timeNow= now(); // unqualified name
\end{cpp}

using声明的应用有以下后果:

\begin{itemize}
\item
若在相同可见性作用域中声明了相同的名称,则会出现查找不明确和编译器错误。

\item
若在周围的可见性作用域中声明了相同的名称,则将被using声明隐藏。
\end{itemize}

\mySamllsection{使用指令}

using指令允许它不加限制地使用所有命名空间名称。

\begin{cpp}
#include <iostream> // either
#include <chrono> // either
...
#import std; // or
...
using namespace std;
...
cout << "Hello world:" << endl; // unqualified name
auto timeNow= chrono::system_clock::now(); // partially qualified name
\end{cpp}

using指令不给当前可见性作用域添加任何名称,只能使名称可访问。所以:

\begin{itemize}
\item
若在相同可见性作用域中声明了相同的名称,则会出现查找不明确和编译器错误。

\item
本地名称空间中的名称隐藏了在周围名称空间中声明的名称。

\item
若在不同的名称空间中可以看到相同的名称,或者名称空间中的名称隐藏了全局作用域中的名称,则会发生歧义查找,从而导致编译器错误。
\end{itemize}

\begin{myWarning}{在源文件中小心使用using指令}

在源文件中使用ssing指令应该非常小心,因为通过使用命名空间std的指令,所有来自std的名字都是可见的。这包括在本地或周围名称空间中意外隐藏名称的名称。

不要在头文件中使用using指令。若包含了一个带有using namespace std指令的头文件,那么std中的所有名字都是可见的。

\end{myWarning}

\mySamllsection{命名空间别名}

命名空间别名定义命名空间的同义词。对于较长的名称空间名称或嵌套的名称空间,使用别名通常很方便

\begin{cpp}
#include <chrono> // either
...
#import std; // or
...
namespace sysClock= std::chrono::system_clock;
auto nowFirst= sysClock::now();
auto nowSecond= std::chrono::system_clock::now();
\end{cpp}

由于命名空间别名的存在,可以使用该别名来处理现在限定的函数。命名空间别名不能隐藏名称。


\mySamllsection{构建可执行文件}

很少需要针对库显式链接,并且是平台相关的。例如,对于当前的g++或clang++编译器,应该链接到pthread库以获得多线程功能。

\begin{shell}
g++ -std=c++14 thread.cpp -o thread -pthread
\end{shell}