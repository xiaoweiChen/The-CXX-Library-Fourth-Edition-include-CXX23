存在许多非成员函数用于操作文件系统。

\begin{center}
用于操作文件系统的非成员函数
\end{center}

% Please add the following required packages to your document preamble:
% \usepackage{longtable}
% Note: It may be necessary to compile the document several times to get a multi-page table to line up properly
\begin{longtable}[c]{|l|l|}
\hline
\textbf{非成员函数}                 & \textbf{描述}                                            \\ \hline
\endfirsthead
%
\endhead
%
absolute                                      & 组成绝对路径。                                      \\ \hline
canonical 和 weakly\_canonical               & 组成规范路径。                                      \\ \hline
relative 和 proximate                        & 组成相对路径。                                       \\ \hline
copy                                          & 复制文件或目录。                                    \\ \hline
copy\_file                                    & 复制文件内容。                                           \\ \hline
copy\_symlink                                 & 复制一个符号链接。                                         \\ \hline
create\_directory and create\_directories     & 创建一个新目录。                                        \\ \hline
create\_hard\_link                            & 创建一个硬链接。                                            \\ \hline
create\_symlink 和 create\_directoy\_symlink & 创建符号链接。                                        \\ \hline
current\_path                                 & 返回当前工作目录。                          \\ \hline
exists                                        & 检查路径是否引用现有文件。                  \\ \hline
equivalent                                    & 检查两个路径是否指向同一个文件。                     \\ \hline
file\_size                                    & 返回文件的大小。                                       \\ \hline
hard\_link\_count                             & 返回文件的硬链接数。                     \\ \hline
last\_write\_time                             & 获取并设置最后一次文件修改的时间。           \\ \hline
premissions                                   & 修改文件访问权限。                           \\ \hline
read\_symlink                                 & 获取符号链接的目标。                           \\ \hline
remove                                        & 删除文件或空目录。                           \\ \hline
remove\_all                                   & 递归地删除文件或目录及其所有内容。 \\ \hline
rename                                        & 移动或重命名文件或目录。                           \\ \hline
resize\_file                                  & 通过截断更改文件的大小。                       \\ \hline
space                                         & 返回文件系统上的可用空间。                       \\ \hline
status                                        & 确定文件属性。                                 \\ \hline
symlink\_status                               & 确定文件属性并检查符号链接目标。  \\ \hline
temp\_directory\_path                         & 返回临时文件的目录。                        \\ \hline
\end{longtable}

\mySamllsection{读取和设置文件的最后写入时间}

通过全局函数std::filesystem::last\_write\_time,可以读取和设置文件的最后写入时间。下面是一个基于\href{http://en.cppreference.com/w/cpp/experimental/fs/last_write_time}{en.cppreference.com}中的last\_write\_time示例的示例。

\filename{文件创建的时间}

\begin{cpp}
// fileTime.cpp
...
#include <filesystem>

...

namespace fs = std::filesystem;
using namespace std::chrono_literals;

...

fs::path path = fs::current_path() / "rainer.txt";
std::ofstream(path.c_str());
const auto fTime = fs::last_write_time(path); // (1)
const auto sTime = std::chrono::clock_cast<std::chrono::system_clock>(fTime); // (2)


std::time_t cftime = std::chrono::system_clock::to_time_t(sTime);
std::cout << "Write time on server " // (3)
		  << std::asctime(std::localtime(&cftime));
std::cout << "Write time on server " // (4)
		  << std::asctime(std::gmtime(&cftime)) << '\n';

const auto fTime2 = fTime + 2h; // (5)
const auto sTime2 = std::chrono::clock_cast<std::chrono::system_clock>(fTime2); // (6)
std::time_t cftime2 = std::chrono::system_clock::to_time_t(sTime2);
std::cout << "Local time on client "
		  << std::asctime(std::localtime(&cftime2)) << '\n';
\end{cpp}

第(1)行给出了新创建文件的写时间,可以使用(2)中的fTime将其转换为一个实时挂钟sTime。这个sTime在下面一行初始化std::chrono::system\_clock。cftime的类型为std::filesystem::file\_time\_type,在本例中,它是std::chrono::system\_clock的别名;可以在(3)中初始化std::localtime,并以文本形式表示日历时间。若使用std::gmtime(4)而不是std::localtime,则不会有任何变化。这让我很困惑,因为协调世界时(UTC)与德国当地时间相差2小时。这是由于服务器的在线编译器在en.cppreference.com,服务器将UTC时间与本地时间设置为相同时间。

下面是程序的输出。我将文件的写入时间移到未来2小时(5),并从文件系统(6)中读取它。这将调整时间,因此它与德国的当地时间相对应。

\myGraphic{0.6}{content/chapter18/images/4.jpg}{}

\mySamllsection{文件系统空间信息}

全局函数std::filesystem::space返回一个std::filesystem::space\_info对象,该对象有三个成员:capacity、free和available。

\begin{itemize}
\item 
capacity: 文件系统的总大小

\item 
free: 文件系统上的空闲空间

\item 
available: 给非特权进程的自由空间(等于或小于自由)
\end{itemize}

所有大小都以字节为单位。

下面程序的输出来自于cppreference.com。这里尝试的所有路径都在同一个文件系统上,因此总是得到相同的答案。

\filename{空间信息}

\begin{cpp}
// space.cpp
...
#include <filesystem>

...

namespace fs = std::filesystem;

...

fs::space_info root = fs::space("/");
fs::space_info usr = fs::space("/usr");

std::cout << ". Capacity Free Available\n"
		  << "/ " << root.capacity << " "
		  << root.free << " " << root.available << "\n"
		  << "usr " << usr.capacity << " "
		  << usr.free << " " << usr.available;
\end{cpp}


\myGraphic{0.6}{content/chapter18/images/5.jpg}{}











