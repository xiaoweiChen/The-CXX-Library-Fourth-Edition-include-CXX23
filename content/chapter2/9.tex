新的C++17数据类型std::any、std::optional和std::variant都基于\href{http://www.boost.org/}{Boost库}。

\mySamllsection{std::any}

\href{http://en.cppreference.com/w/cpp/utility/any}{std::any}为任何可复制构造类型的单个值的类型安全容器,该类型需要头文件<any>。有几种方法可以创建std::any容器,可以使用各种构造函数或工厂函数std::make\_any。any.emplace允许将一个值构造为任何值,any.reset能够销毁所包含的对象。若想知道容器std::any是否有值,请使用成员函数any.has\_value,也可以通过any.type获得容器对象的类型id。因为泛型函数std::any\_cast,可以访问所包含的对象。若指定了错误的类型,就会得到std::bad\_any\_cast异常。下面的代码片段显示了std::any的基本用法。

\filename{std::any}

\begin{cpp}
// any.cpp
...
#include <any>

struct MyClass{};

...

std::vector<std::any> anyVec{true, 2017, std::string("test"), 3.14, MyClass()};
std::cout << std::any_cast<bool>(anyVec[0]); // true
int myInt= std::any_cast<int>(anyVec[1]);
std::cout << myInt << '\n'; // 2017

std::cout << anyVec[0].type().name(); // b
std::cout << anyVec[1].type().name(); // i
\end{cpp}

这里定义了一个std::vector<std::any>。要获得其元素,必须使用std::any\_cast。若使用了错误的类型,就会得到std::bad\_any\_cast异常。

\begin{myNotic}{typeid的字符串表示形式}
类型标识符的字符串表示形式是实现定义的。若anyVec[1]是int类型,则表达式anyVec[1].type().name()将使用\href{https://gcc.gnu.org/}{GCC C++编译器}返回i,使用\href{https://en.wikipedia.org/wiki/Microsoft_Visual_C%2B%2B}{Microsoft Visual C++编译器}返回int。
\end{myNotic}

std::any可以拥有任意类型的对象;std::optional可以有值,也可以没有值。

\mySamllsection{std::optional}

\href{http://en.cppreference.com/w/cpp/utility/optional}{std::optional}对于可能有结果的数据库查询等计算来说非常合适,这个类型需要头文件<optional>。

\begin{myTip}{不要定义“无结果”}
C++17之前,通常使用惟一值(如空指针、空字符串或惟一整数)来表示结果的缺失。对于类型系统,必须使用常规值(例如空字符串)来定义不规则值。这些唯一的值或没有结果很容易出错,因为必须用类型系统来检查返回值。
\end{myTip}

各种构造函数和方便的函数std::make\_optional允许定义一个std::optional对象opt,带或不带值。opt.emplace将在原地构造包含的值,而opt.reset将销毁容器值。可以显式地询问std::optional容器是否有值,或者可以在逻辑表达式中检查它。opt.value返回值,而opt.value\_or返回值或默认值。若opt没有包含值,opt.value将抛出std::bad\_optional\_access异常。

下面是一个使用std::optional的简短示例。

\filename{std::optional}

\begin{cpp}
// optional.cpp
...
#include <optional>

std::optional<int> getFirst(const std::vector<int>& vec){
	if (!vec.empty()) return std::optional<int>(vec[0]);
	else return std::optional<int>();
}

...

std::vector<int> myVec{1, 2, 3};
std::vector<int> myEmptyVec;

auto myInt= getFirst(myVec);

if (myInt){
	std::cout << *myInt << '\n'; // 1
	std::cout << myInt.value() << '\n'; // 1
	std::cout << myInt.value_or(2017) << '\n'; // 1
}

auto myEmptyInt= getFirst(myEmptyVec);

if (!myEmptyInt){
	std::cout << myEmptyInt.value_or(2017) << '\n'; // 2017
}
\end{cpp}

函数getFirst中使用std::optional。若第一个元素存在,getFirst返回该元素。若没有,将得到一个std::optional<int>对象。主函数有两个vector。两者都调用getFirst并返回一个std::optional对象。在myInt的情况下,对象有一个值;myEmptyInt的情况下,该对象没有值。程序显示myInt和myEmptyInt的值。myInt.value\_or(2017)返回值,但myEmptyInt.value\_or(2017)会返回默认值。

C++23中,std::optional扩展为一元操作opt.and\_then、opt.transform和opt.or\_else。and\_then返回给定函数调用的结果(若存在),或者返回一个空的std::optional。transform返回一个std::optional,其中包含转换后的值,或者一个空std::optional。此外,opt.or\_else若包含值或给定函数的结果,则返回std::optional。

这些一元操作支持std::optional上的复合操作:

\filename{std::optional上的一元操作}

\begin{cpp}
// optionalMonadic.cpp

#include <iostream>
#include <optional>
#include <vector>
#include <string>

std::optional<int> getInt(std::string arg) {
	try {
		return {std::stoi(arg)};
	}
	catch (...) {
		return { };
	}
}


int main() {

	std::vector<std::optional<std::string>> strings = {"66", "foo", "-5"};
	
	for (auto s: strings) {
		auto res = s.and_then(getInt)
					.transform( [](int n) { return n + 100;})
					.transform( [](int n) { return std::to_string(n); })
					.or_else([] { return std::optional{std::string("Error") }; });
		std::cout << *res << ' '; // 166 Error 95
	}

}
\end{cpp}

基于范围的for循环(第22行)遍历std::vector<std::optional<std::string>{}>。getInt函数将每个元素转换为整数(第23行),为其添加100(第24行),将其转换回字符串(第25行),最后显示字符串(第27行)。若初始转换为int失败,则返回字符串Error(第26行)并显示。

\mySamllsection{std::variant}

\href{http://en.cppreference.com/w/cpp/utility/variant}{std::variant}是类型安全的联合,使用该类型需要头文件<variant>。std::variant的实例具有来自其类型之一的值,类型不能是引用、数组或void。std::variant可以有多个类型,默认初始化的std::variant使用其第一个类型初始化,其第一个类型必须有一个默认构造函数。通过使用var.index,可以获得std::variant var持有的备选项的从零开始的索引。若该变量持有值,则var.valueless\_by\_exception返回false,可以使用var.emplace在原地创建一个新值。一些全局函数用于访问std::variant。函数模板var.holds\_alternative允许检查std::variant是否包含指定的替代,可以将std::get与索引和类型作为参数一起使用。通过使用索引,将获得该值。若使用类型调用std::get,则只有在该值唯一的情况下才会获得该值。若使用无效索引或非唯一类型,则会得到std::bad\_variant\_access异常。std::get最终返回一个异常,而std::get\_if在发生错误时,会返回一个空指针。

下面的代码片段展示了std::variant的用法。

\filename{std::variant}

\begin{cpp}
// variant.cpp
...
#include <variant>

...

std::variant<int, float> v, w;
v = 12; // v contains int
int i = std::get<int>(v);
w = std::get<int>(v);
w = std::get<0>(v); // same effect as the previous line
w = v; // same effect as the previous line

// std::get<double>(v); // error: no double in [int, float]
// std::get<3>(v); // error: valid index values are 0 and 1

try{
	std::get<float>(w); // w contains int, not float: will throw
}
catch (std::bad_variant_access&) {}

std::variant<std::string> v2("abc"); // converting constructor must be unambiguous
v2 = "def"; // converting assignment must be unambiguous
\end{cpp}

v和w是两个std::variant对象,两者都可以有int和float值,默认值为0。v为12时,下面的调用std::get<int>(V)返回值。接下来的三行显示了将变量v赋值给w的三种可能性,但必须记住一些规则。可以通过类型std::get<double>(v)或index: std::get<3>(v)来查询变量值。类型必须唯一且索引有效,变量w保存一个int值。若为浮点类型,就会得到std::bad\_variant\_access异常。若构造函数调用或赋值调用是明确的,则可以进行转换。这使得从C-string构造std::variant<std::string>,或为该变体赋值一个新的C-string成为可能。

std::variant有一个有趣的非成员函数std::visit,允许在变量列表上执行可调用对象。可调用对象是自定义的可调用对象。通常,这可以是一个函数、函数对象或Lambda表达式。为简单起见,在本例中使用Lambda函数。

\filename{std::visit}

\begin{cpp}
// visit.cpp
...
#include <variant>

...

std::vector<std::variant<char, long, float, int, double, long long>>
			vecVariant = {5, '2', 5.4, 100ll, 2011l, 3.5f, 2017};
			
for (auto& v: vecVariant){
	std::visit([](auto&& arg){std::cout << arg << " ";}, v);
	// 5 2 5.4 100 2011 3.5 2017
}

// display each type
for (auto& v: vecVariant){
	std::visit([](auto&& arg){std::cout << typeid(arg).name() << " ";}, v);
	// int char double __int64 long float int
}

// get the sum
std::common_type<char, long, float, int, double, long long>::type res{};

std::cout << typeid(res).name() << '\n'; // double

for (auto& v: vecVariant){
	std::visit([&res](auto&& arg){res+= arg;}, v);
}
std::cout << "res: " << res << '\n'; // 4191.9

// double each value
for (auto& v: vecVariant){
	std::visit([&res](auto&& arg){arg *= 2;}, v);
	std::visit([](auto&& arg){std::cout << arg << " ";}, v);
	// 10 d 10.8 200 4022 7 4034
}
\end{cpp}

本例中的每个variant实例都可以保存char、long、float、int、double或long long类型。第一个访问器[](auto\&\& arg)\{std::cout <{}< arg <{}< " ";\}将输出各种变量。第二个访问器std::cout <{}< typeid(arg).name() <{}< " ";\}将显示其类型。

现在我想对这些变量的元素乘2,在编译时需要正确的结果类型。来自类型特征库的std::common\_type将其提供。common\_type给出了所有char、long、float、int、double和long long类型都可以隐式转换的类型。res\{\}中的最后一个\{\}导致其初始化为0.0,Res是double类型。访问者[\&res](auto\&\& arg)\{arg *= 2;\}计算总和,下面一行显示相应结果。



















