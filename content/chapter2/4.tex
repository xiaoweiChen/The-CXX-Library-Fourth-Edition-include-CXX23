You can create tuples of arbitrary length and types with \href{http://en.cppreference.com/w/cpp/utility/tuple}{std::tuple}. The class template needs the header <tuple>. std::tuple is a generalization of std::pair. Tuples with two elements and pairs can be converted into each other. Like std::pair, the tuple has a default, a copy, and a move constructor.

You can swap tuples with the function std::swap.
std::get can access the i-th element of a tuple: std::get<i-1>(t). By std::get<type>(t), you can directly refer to the element of the type type.

Tuples support the comparison operators ==, !=, <, >, <= and >=. If you compare two tuples, the elements of the tuples are compared lexicographically. The comparison starts at the index 0.

\mySamllsection{std::make\_tuple}

The helper function \href{http://en.cppreference.com/w/cpp/utility/tuple/make_tuple}{std::make\_tuple} provides a convenient way to create a tuple. You don’t have to specify the types. The compiler automatically deduces them.

\filename{The helper function std::make\_tuple}

\begin{cpp}
// tuple.cpp
...
#include <tuple>
...
using std::get;

std::tuple<std::string, int, float> tup1("first", 3, 4.17f);
auto tup2= std::make_tuple("second", 4, 1.1);

std::cout << get<0>(tup1) << ", " << get<1>(tup1) << ", "
		  << get<2>(tup1) << '\n'; // first, 3, 4.17
std::cout << get<0>(tup2) << ", " << get<1>(tup2) << ", "
		  << get<2>(tup2) << '\n'; // second, 4, 1.1
std::cout << (tup1 < tup2) << '\n'; // true

get<0>(tup2)= "Second";
std::cout << get<0>(tup2) << "," << get<1>(tup2) << ","
		  << get<2>(tup2) << '\n'; // Second, 4, 1.1
std::cout << (tup1 < tup2) << '\n'; // false

auto pair= std::make_pair(1, true);
std::tuple<int, bool> tup= pair;
\end{cpp}

\mySamllsection{std::tie and std::ignore}

\href{http://en.cppreference.com/w/cpp/utility/tuple/tie}{std::tie} enables you to create tuples that reference variables. You can explicitly ignore tuple elements with \href{http://en.cppreference.com/w/cpp/utility/tuple/ignore}{std::ignore}.

\filename{The helper functions std::tie and std::ignore}

\begin{cpp}
// tupleTie.cpp
...
#include <tuple>
...
using namespace std;

int first= 1;
int second= 2;
int third= 3;

int fourth= 4;
cout << first << " " << second << " "
	 << third << " " << fourth << endl; // 1 2 3 4

auto tup= tie(first, second, third, fourth) // bind the tuple
		= std::make_tuple(101, 102, 103, 104); // create the tuple
												// and assign it

cout << get<0>(tup) << " " << get<1>(tup) << " " << get<2>(tup)
	 << " " << get<3>(tup) << endl; // 101 102 103 104
cout << first << " " << second << " " << third << " "
	 << fourth << endl; // 101 102 103 104

first= 201;
get<1>(tup)= 202;
cout << get<0>(tup) << " " << get<1>(tup) << " " << get<2>(tup)
	 << " " << get<3>(tup) << endl; // 201 202 103 104
cout << first << " " << second << " " << third << " "
	 << fourth << endl; // 201 202 103 104

int a, b;
tie(std::ignore, a, std::ignore, b)= tup;
cout << a << " " << b << endl; // 202 104
\end{cpp}








