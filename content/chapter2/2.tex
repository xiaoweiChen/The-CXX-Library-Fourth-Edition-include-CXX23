
The three functions std::bind, std::bind\_front, std::bind\_back, and std::functions fit very well together. While std::bind, std::bind\_front, or std::bind\_back enables you to create new function objects on the fly, std::function takes these temporary function objects and binds them to a variable.

\begin{myTip}{std::bind, std::bind\_front, std::bind\_back, and std::function are mostly superfluous}
	
std::bind and std::function, which were part of \href{https://en.wikipedia.org/wiki/C%2B%2B_Technical_Report_1}{TR1}, std::bind\_front, and std::bind\_back are mostly not necessary in C++. First, you can use lambdas instead of std::bind, std::bind\_front, std::bind\_back, and second, you can most often use the automatic type deduction with auto instead of std::function.
	
\end{myTip}

\mySamllsection{std::bind}

The function std::bind is more powerful than std::bind\_front or std::bind\_back because std::bind allows the arguments to be bound to an arbitrary position. While std::bind, std::bind\_front, and std::bind\_back enables you to create new function objects on the fly, std::function takes these temporary function objects and binds them to a variable. These functions are powerful tools from functional programming and require the header <functional>.

\filename{Creating and binding function objects}

\begin{cpp}
// bindAndFunction.cpp
...
#include <functional>
...
// for placehoder _1 and _2
using namespace std::placeholders;

using std::bind;
using std::bind_front;
using std::bind_back;
using std::function
...
double divMe(double a, double b){ return a/b; };

function<double(double, double)> myDiv1 = bind(divMe, _1, _2); // 200
function<double(double)> myDiv2 = bind(divMe, 2000, _1); // 200
function<double(double>> myDiv3 = bind_front(divMe, 2000); // 200
function<double(double)> myDiv4 = bind_back(divMe, 10); // 200
\end{cpp}

Because of \href{http://en.cppreference.com/w/cpp/utility/functional/bind}{std::bind}, you can create function objects in a variety of ways:

\begin{itemize}
\item 
bind the arguments to an arbitrary position,

\item 
change the order of the arguments,

\item 
introduce placeholders for arguments,

\item 
partially evaluate functions,

\item 
invoke the newly created function objects, use them in the STL algorithm, or store them in std::function.
\end{itemize}

\mySamllsection{std::bind\_front (C++20)}

\href{https://en.cppreference.com/w/cpp/utility/functional/bind_front}{std::bind\_front} creates a callable wrapper from a callable. A call std::bind\_front(func, arg ...) binds all arguments arg to the front of func and returns a callable wrapper.


\mySamllsection{std::bind\_back (C++23)}

\href{https://en.cppreference.com/w/cpp/utility/functional/bind_front}{std::bind\_back} creates a callable wrapper from a callable. A call std::bind\_back(func, arg ...) binds all arguments arg to the back of func and returns a callable wrapper.

\mySamllsection{std::function}

\href{http://en.cppreference.com/w/cpp/utility/functional/function}{std::function} can store arbitrary callables in variables. It’s a polymorphic function wrapper. A callable may be a lambda function, a function object, or a function. std::function is always necessary and can’t be replaced by auto, if you have to specify the type of the callable explicitly.

\filename{A dispatch table with std::function}

\begin{cpp}
// dispatchTable.cpp
...
#include <functional>
...
using std::make_pair;
using std::map;

map<const char, std::function<double(double, double)>> tab;
tab.insert(make_pair('+', [](double a, double b){ return a + b; }));
tab.insert(make_pair('-', [](double a, double b){ return a - b; }));
tab.insert(make_pair('*', [](double a, double b){ return a * b; }));
tab.insert(make_pair('/', [](double a, double b){ return a / b; }));

std::cout << tab['+'](3.5, 4.5); // 8
std::cout << tab['-'](3.5, 4.5); // -1
std::cout << tab['*'](3.5, 4.5); // 15.75
std::cout << tab['/'](3.5, 4.5); // 0.777778
\end{cpp}

The type parameter of std::function defines the type of callables std::function will accept.

\begin{center}
Return type and type of the arguments
\end{center}

% Please add the following required packages to your document preamble:
% \usepackage{longtable}
% Note: It may be necessary to compile the document several times to get a multi-page table to line up properly
\begin{longtable}[c]{|l|l|l|}
\hline
Function type          & Return type & Type of the arguments \\ \hline
\endfirsthead
%
\endhead
%
double(double, double) & double      & double                \\ \hline
int()                  & int         &                       \\ \hline
double(int, double)    & double      & int, double           \\ \hline
void()                 &             &                       \\ \hline
\end{longtable}






















