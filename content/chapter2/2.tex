
std::bind、std::bind\_front、std::bind\_back和std::functions这三个函数可以组合得非常好。std::bind、std::bind\_front或std::bind\_back能够动态地创建新的函数对象,而std::function接受这些临时函数对象并将它们绑定到一个变量。

\begin{myTip}{std::bind, std::bind\_front, std::bind\_back和std::function常常是多余的}
	
std::bind和std::function,是\href{https://en.wikipedia.org/wiki/C%2B%2B_Technical_Report_1}{TR1},std::bind\_front和std::bind\_back的一部分,在C++中大多是不必要的。首先,可以使用Lambda来代替std::bind、std::bind\_front、std::bind\_back;其次,可以经常使用auto来代替std::function进行自动类型推断。
	
\end{myTip}

\mySamllsection{std::bind}

函数std::bind比std::bind\_front或std::bind\_back更强大,因为std::bind允许将参数绑定到任意位置。std::bind、std::bind\_front和std::bind\_back能够动态地创建新的函数对象,而std::function接受这些临时函数对象并将它们绑定到一个变量。这些函数是函数式编程的强大工具,并且需要头文件<functional>。

\filename{创建和绑定函数对象}

\begin{cpp}
// bindAndFunction.cpp
...
#include <functional>
...
// for placehoder _1 and _2
using namespace std::placeholders;

using std::bind;
using std::bind_front;
using std::bind_back;
using std::function
...
double divMe(double a, double b){ return a/b; };

function<double(double, double)> myDiv1 = bind(divMe, _1, _2); // 200
function<double(double)> myDiv2 = bind(divMe, 2000, _1); // 200
function<double(double>> myDiv3 = bind_front(divMe, 2000); // 200
function<double(double)> myDiv4 = bind_back(divMe, 10); // 200
\end{cpp}

\href{http://en.cppreference.com/w/cpp/utility/functional/bind}{std::bind}可以用多种方式创建函数对象:

\begin{itemize}
\item 
将参数绑定到任意位置,

\item 
改变参数的顺序,

\item 
为参数引入占位符,

\item 
对函数部分求值,

\item 
调用新创建的函数对象,在STL算法中使用它们,或者将其存储在std::function中。
\end{itemize}

\mySamllsection{std::bind\_front (C++20)}

\href{https://en.cppreference.com/w/cpp/utility/functional/bind_front}{std::bind\_front}从可调用对象创建可调用包装器。调用std::bind\_front(func, arg…)将所有参数arg绑定到func的前端,并返回一个可调用的包装器。


\mySamllsection{std::bind\_back (C++23)}

\href{https://en.cppreference.com/w/cpp/utility/functional/bind_front}{std::bind\_back} 从可调用对象创建可调用包装器。调用std::bind\_back(func, arg…)将所有参数arg绑定到func的后面,并返回一个可调用的包装器。

\mySamllsection{std::function}

\href{http://en.cppreference.com/w/cpp/utility/functional/function}{std::function}可以在变量中存储可调用对象,是一个多态函数包装器。可调用对象可以是Lambda函数、函数对象或函数。若必须显式指定可调用对象的类型,则std::function是必需的,不能用auto取代。

\filename{使用std::function的调度表}

\begin{cpp}
// dispatchTable.cpp
...
#include <functional>
...
using std::make_pair;
using std::map;

map<const char, std::function<double(double, double)>> tab;
tab.insert(make_pair('+', [](double a, double b){ return a + b; }));
tab.insert(make_pair('-', [](double a, double b){ return a - b; }));
tab.insert(make_pair('*', [](double a, double b){ return a * b; }));
tab.insert(make_pair('/', [](double a, double b){ return a / b; }));

std::cout << tab['+'](3.5, 4.5); // 8
std::cout << tab['-'](3.5, 4.5); // -1
std::cout << tab['*'](3.5, 4.5); // 15.75
std::cout << tab['/'](3.5, 4.5); // 0.777778
\end{cpp}

函数的类型形参定义了std::function接受的可调用对象的类型。

\begin{center}
返回类型和参数类型
\end{center}

% Please add the following required packages to your document preamble:
% \usepackage{longtable}
% Note: It may be necessary to compile the document several times to get a multi-page table to line up properly
\begin{longtable}[c]{|l|l|l|}
\hline
函数类型          & 返回类型 & 参数类型 \\ \hline
\endfirsthead
%
\endhead
%
double(double, double) & double      & double                \\ \hline
int()                  & int         &                       \\ \hline
double(int, double)    & double      & int, double           \\ \hline
void()                 &             &                       \\ \hline
\end{longtable}






















