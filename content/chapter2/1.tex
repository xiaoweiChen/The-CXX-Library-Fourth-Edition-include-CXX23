可以将min、max和minmax函数的许多变体应用于值和初始化列表。这些函数需要头文件<algorithm>。相反,std::move、std::forward、std::to\_underlying和std::swap函数在头文件<utility>中定义。

\mySamllsection{std::min, std::max和std::minmax}

在header <algorithm>中定义的函数\href{http://en.cppreference.com/w/cpp/algorithm/min}{std::min}、\href{http://en.cppreference.com/w/cpp/algorithm/max}{std::max}和\href{http://en.cppreference.com/w/cpp/algorithm/minmax}{std::minmax}作用于值和初始化器列表,并返回所请求的值作为结果。在std::minmax的情况下,了得到一个std::pair。这对的第一个元素是最小值;第二个是值的最大值。默认情况下使用less操作符(<),但您可以应用比较操作符。这个函数需要两个参数并返回一个布尔值。返回true或false的函数称为谓词。

\mySamllsection{std::min, std::max和std::minmax}

\begin{cpp}
// minMax.cpp
...
#include <algorithm>
...
using std::cout;
...
cout << std::min(2011, 2014); // 2011
cout << std::min({3, 1, 2011, 2014, -5}); // -5
cout << std::min(-10, -5, [](int a, int b)
                { return std::abs(a) < std::abs(b); }); // -5

auto pairInt= std::minmax(2011, 2014);
auto pairSeq= std::minmax({3, 1, 2011, 2014, -5});
auto pairAbs= std::minmax({3, 1, 2011, 2014, -5}, [](int a, int b)
                      { return std::abs(a) < std::abs(b); });

cout << pairInt.first << "," << pairInt.second; // 2011,2014
cout << pairSeq.first << "," << pairSeq.second; // -5,2014
cout << pairAbs.first << "," << pairAbs.second; // 1,2014
\end{cpp}

该表提供了std::min、std::max和std::minmax函数的概述

\begin{center}
std::min、std::max和std::minmax的不同
\end{center}

% Please add the following required packages to your document preamble:
% \usepackage{longtable}
% Note: It may be necessary to compile the document several times to get a multi-page table to line up properly
\begin{longtable}[c]{|l|l|}
\hline
\textbf{函数}           & \textbf{描述}                                                                \\ \hline
\endfirsthead
%
\endhead
%
min(a, b)                   & 返回a和b中较小的值                                                \\ \hline
min(a, b, comp)             & 根据谓词comp返回a和b的较小值。               \\ \hline
min(initialiser list)       & 返回初始化列表的最小值。                                 \\ \hline
min(initialiser list, comp) & 根据谓词comp返回初始化项列表的最小值。 \\ \hline
max(a, b)                   & 返回a和b中较大的值。                                               \\ \hline
max(a, b, comp)             & 根据谓词comp返回a和b中较大的值。               \\ \hline
max(initialiser list)       & 返回初始化项列表的最大值。                                 \\ \hline
max(initialiser list, comp) & 根据谓词comp返回初始化项列表的最大值。 \\ \hline
minmax(a, b)                & 返回a和b的较大和较小的值。                                   \\ \hline
minmax(a, b, comp)          & 根据谓词comp返回a和b的较大和较小的值。   \\ \hline
minmax(initialiser list)    & 返回初始化项列表的最小值和最大值。                \\ \hline
minmax(initialiser list, comp) & 根据谓词comp返回初始化项列表的最小值和最大值。 \\ \hline
\end{longtable}

\mySamllsection{std::midpoint和std::lerp}

函数std::midpoint(a, b)计算a和b之间的中点。a和b可以是整数、浮点数或指针。若a和b是指针,必须指向同一个数组对象。函数std::midpoint要求包含头文件<numeric>。

函数std::lerp(a, b, t)计算两个数字的线性插值,需要头文件<cmath>。返回值是a + t(b - a)。


\mySamllsection{std::midpoint和std::lerp}

\begin{cpp}
// midpointLerp.cpp

#include <cmath>
#include <numeric>

...

std::cout << "std::midpoint(10, 20): " << std::midpoint(10, 20) << '\n';

for (auto v: {0.0, 0.1, 0.2, 0.3, 0.4, 0.5, 0.6, 0.7, 0.8, 0.9, 1.0}) {
	std::cout << "std::lerp(10, 20, " << v << "): " << std::lerp(10, 20, v) << "\n";
}
\end{cpp}


\myGraphic{0.4}{content/chapter2/images/2.jpg}{}

\mySamllsection{std::cmp\_equal, std::cmp\_not\_equal, std::cmp\_less, std::cmp\_greater, std::cmp\_less\_equal和std::cmp\_greater\_equal}


头文件<utility>中定义的函数std::cmp\_equal、std::cmp\_not\_equal、std::cmp\_less、std::cmp\_greater、std::cmp\_less\_equal和std::cmp\_greater\_equal提供了整数的安全比较。安全比较对负符号整数的比较小于对无符号整数的比较,并且对有符号整数或无符号整数以外的值的比较,会产生编译时错误。


\begin{myWarning}{使用内置整数进行整数转换}
下面的代码片段举例说明了有符号/无符号比较的问题。

\begin{cpp}
-1 < 0u; // true
std::cmp_greater( -1, 0u); // false
\end{cpp}

作为有符号整数的-1会提升为无符号类型,会导致一个令人意外的结果。

\end{myWarning}

\mySamllsection{std::move}

头文件<utility>中定义了函数\href{http://en.cppreference.com/w/cpp/utility/move}{std::move},授权编译器移动资源。移动语义中,源对象的值移动到新对象中。之后,源处于定义良好但不确定的状态。大多数情况下,这是源的默认状态。通过使用std::move,编译器将源参数转换为右值引用:static\_cast<std::remove\_reference<decltype(arg)>::type\&\&>(arg)。若编译器不能应用移动语义,则会退回到复制语义:

\begin{cpp}
#include <utility>
...
std::vector<int> myBigVec(10000000, 2011);
std::vector<int> myVec;

myVec = myBigVec; // copy semantics
myVec = std::move(myBigVec); // move semantics
\end{cpp}


\begin{myTip}{移动比复制廉价}
	
移动语义有两个优点。首先,使用便宜的移动而不是昂贵的复制通常是一个好主意。因此不需要额外的内存分配和释放。其次,有些对象是不能复制的。例如,线程或锁。

\end{myTip}

\mySamllsection{std::forward}

头文件<utility>中定义的函数\href{http://en.cppreference.com/w/cpp/utility/forward}{std::forward}能够编写函数模板,这些函数模板可以相同地转发它们的参数。std::forward的典型用例是工厂函数或构造函数。工厂函数创建一个对象,因此应该不加修改地传递其参数。构造函数通常使用实参来初始化具有相同实参的基类。所以std::forward对于泛型库作者来说是一个完美的工具:

\filename{完美转发}

\begin{cpp}
// forward.cpp
...
#include <utility>
...
using std::initialiser_list;

struct MyData{
	MyData(int, double, char){};
};

template <typename T, typename... Args>
T createT(Args&&... args){
	return T(std::forward<Args>(args)... );
}

...

int a= createT<int>();
int b= createT<int>(1);

std::string s= createT<std::string>("Only for testing.");
MyData myData2= createT<MyData>(1, 3.19, 'a');

typedef std::vector<int> IntVec;
IntVec intVec= createT<IntVec>(initialiser_list<int>({1, 2, 3}));
\end{cpp}

函数模板createT必须以\href{https://isocpp.org/blog/2012/11/universal-references-in-c11-scott-meyers}{通用引用}: args\&\&…args的形式接受参数。通用引用或转发引用是类型演绎上下文中的右值引用。

\begin{myTip}{std::forward支持与可变模板结合使用的完全泛型函数}
	
若将std::forward与可变模板一起使用,则可以定义完全泛型的函数模板。函数模板可以接受任意数量的参数,并且不加修改地对其进行转发。
	
\end{myTip}

\mySamllsection{std::to\_underlying}

C++23中的std::to\_underlying函数将枚举enum转换为其底层类型Enum。这个函数是表达式static\_cast<std::underlying\_type<Enum>::type>(Enum)的方便函数,使用类型特征函数std::underlying\_type。


\mySamllsection{std::swap}

通过<utility>中定义的函数\href{http://en.cppreference.com/w/cpp/algorithm/swap}{std::swap},可以轻松地交换两个对象。C++标准库中的泛型实现会在内部使用std::move函数。

\filename{std::swap中使用移动语义}

\begin{cpp}
// swap.cpp
...
#include <utility>
...
template <typename T>
inline void swap(T& a, T& b) noexcept {
	T tmp(std::move(a));
	a = std::move(b);
	b = std::move(tmp);
}
\end{cpp}








































