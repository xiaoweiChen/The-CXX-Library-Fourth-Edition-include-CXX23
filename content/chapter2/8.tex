The \href{http://en.cppreference.com/w/cpp/header/chrono}{time library} consists of the three main components, time point, time duration, and clock. Additionally, the library provides the time of day functionality, calendar support, time zone support, and support for in- and output.

\noindent
Time point 

Time point is defined by a starting point, the so-called epoch, and additional time duration.

\noindent
Time duration 

Time duration is the difference between two time-points. The number of ticks gives it.

\noindent
Clock 
A clock consists of a starting point (epoch) and a tick so that the current time point can be calculated.

\noindent
Time of day

The duration since midnight split into hours:minutes:seconds.

\noindent
Calendar 

Calendar stands for various calendar days such as year, a month, a weekday, or the n-th day of a week.

\noindent
Time zone 

Represents time specific to a geographic area.

\begin{myNotic}{The time library is a key component for multithreading}
	
The time library is a key component of the new multithreading capabilities of C++. You can put the current thread by std::this\_thread::sleep\_for(std::chrono::milliseconds(15) for 15 milliseconds to sleep, or you try to acquire a lock for 2 minutes: lock.try\_lock\_until(now + std::chrono::minutes(2)).
	
\end{myNotic}


\mySamllsection{Time Point}

A duration consists of a period of time, defined as some number of ticks of some time unit. A time point consists of a clock and a time duration. This time duration can be positive or negative.

\begin{cpp}
template <class Clock, class Duration= typename Clock::duration> class time_point;
\end{cpp}

The epoch is not defined for the clocks std::chrono::steady\_clock, std::chrono::high\_resolution\_clock and std::chrono::system. But on the popular platform, the epoch of std::chrono::system is usually defined as 1.1.1970. You can calculate the time since 1.1.1970 in the resolutions nanoseconds, seconds and minutes.

\filename{Time since epoch}

\begin{cpp}
// epoch.cpp
...
#include <chrono>
...
auto timeNow= std::chrono::system_clock::now();
auto duration= timeNow.time_since_epoch();
std::cout << duration.count() << "ns" // 1413019260846652ns

typedef std::chrono::duration<double> MySecondTick;
MySecondTick mySecond(duration);
std::cout << mySecond.count() << "s"; // 1413019260.846652s

const int minute= 60;
typedef std::chrono::duration<double, <minute>> MyMinuteTick;
MyMinuteTick myMinute(duration);
std::cout << myMinute.count() << "m"; // 23550324.920572m
\end{cpp}

Thanks to the function std::chrono::clock\_cast, you can cast time points between various clocks.

\begin{myTip}{Easy performance tests with the time library}
	
\filename{Performance measurement}

\begin{cpp}
// performanceMeasurement.cpp
...
#include <chrono>
...
std::vector<int> myBigVec(10000000, 2011);
std::vector<int> myEmptyVec1;

auto begin= std::chrono::high_resolution_clock::now();
myEmptyVec1 = myBigVec;
auto end= std::chrono::high_resolution_clock::now() - begin;

auto timeInSeconds = std::chrono::duration<double>(end).count();
std::cout << timeInSeconds << '\n'; // 0.0150688800
\end{cpp}
	
\end{myTip}


\mySamllsection{Time Duration}

Time duration is the difference between the two time-points. Time duration is measured in the number of ticks.

\begin{cpp}
template <class Rep, class Period = ratio<1>> class duration;
\end{cpp}

If Rep is a floating-point number, the time duration supports fractions of ticks. The most important time durations are predefined in the chrono library:


\begin{cpp}
typedef duration<signed int, nano> nanoseconds;
typedef duration<signed int, micro> microseconds;
typedef duration<signed int, milli> milliseconds;
typedef duration<signed int> seconds;
typedef duration<signed int, ratio< 60>> minutes;
typedef duration<signed int, ratio<3600>> hours;
\end{cpp}

How long can a time duration be? The C++ standard guarantees that the predefined time durations can store +/- 292 years. You can easily define your time duration like a German school hour: typedef std::chrono::duration<double, std::ratio<2700>> MyLessonTick. Time durations in natural numbers must be explicitly converted to time durations in floating pointer numbers. The value will be truncated:

\filename{Durations}

\begin{cpp}
// duration.cpp
...
#include <chrono>
#include <ratio>

using std::chrono;

typedef duration<long long, std::ratio<1>> MySecondTick;
MySecondTick aSecond(1);

milliseconds milli(aSecond);
std::cout << milli.count() << " milli"; // 1000 milli

seconds seconds(aSecond);
std::cout << seconds.count() << " sec"; // 1 sec

minutes minutes(duration_cast<minutes>(aSecond));
std::cout << minutes.count() << " min"; // 0 min

typedef duration<double, std::ratio<2700>> MyLessonTick;
MyLessonTick myLesson(aSecond);
std::cout << myLesson.count() << " less"; // 0.00037037 less
\end{cpp}

\begin{myNotic}{std::ratio}
	
std::ratio supports arithmetic at compile time with rational numbers. A rational number has two template arguments: the nominator and the denominator. C++11 predefines lots of rational numbers.
	
\begin{cpp}
typedef ratio<1, 1000000000000000000> atto;
typedef ratio<1, 1000000000000000> femto;
typedef ratio<1, 1000000000000> pico;
typedef ratio<1, 1000000000> nano;
typedef ratio<1, 1000000> micro;
typedef ratio<1, 1000> milli;
typedef ratio<1, 100> centi;
typedef ratio<1, 10> deci;
typedef ratio< 10, 1> deca;
typedef ratio< 100, 1> hecto;
typedef ratio< 1000, 1> kilo;
typedef ratio< 1000000, 1> mega;
typedef ratio< 1000000000, 1> giga;
typedef ratio< 1000000000000, 1> tera;
typedef ratio< 1000000000000000, 1> peta;
typedef ratio< 1000000000000000000, 1> exa;
\end{cpp}
	
\end{myNotic}

C++14 has built-in literals for the most used time durations.

\begin{center}
Built-in literals for time durations
\end{center}

% Please add the following required packages to your document preamble:
% \usepackage{longtable}
% Note: It may be necessary to compile the document several times to get a multi-page table to line up properly
\begin{longtable}[c]{lll}
\textbf{Type}             & \textbf{Suffix} & \textbf{Example} \\
\endfirsthead
%
\endhead
%
std::chrono::hours        & h               & 5h               \\
std::chrono::minutes      & min             & 5min             \\
std::chrono::seconds      & s               & 5s               \\
std::chrono::milliseconds & ms              & 5ms              \\
std::chrono::microseconds & us              & 5us              \\
std::chrono::nanoseconds  & ns              & 5ns             
\end{longtable}

\mySamllsection{Clock}

The clock consists of a starting point and a tick. You can get the current time with the member function now.

\noindent
std::chrono::system\_clock 

System time, which you can synchronize with the external clock.

\noindent
std::chrono::steady\_clock 

Clock, which can not be adjusted.

\noindent
std::chrono::high\_resolution\_clock 

System time with the greatest accuracy.

std::chrono::system\_clock refers typically to the 1.1.1970. You can not adjust std::steady\_clock forward or backward opposite to two other clocks. The member functions to\_time\_t and from\_time\_t can be used to convert between std::chrono::system\_clock and std::time\_t objects.

\mySamllsection{Time of Day}

std::chrono::time\_of\_day splits the duration since midnight into hours:minutes:seconds. The functions std::chrono::is\_am and std::chrono::is\_pm checks if the time is before midday (ante meridiem) or after midday (post meridiem).

A std::chrono::time\_of\_day object tOfDay supports various member functions.

\begin{center}
Member functions of std::chrono::time\_of\_day
\end{center}

% Please add the following required packages to your document preamble:
% \usepackage{longtable}
% Note: It may be necessary to compile the document several times to get a multi-page table to line up properly
\begin{longtable}[c]{|l|l|}
\hline
\textbf{Member function} & \textbf{Description}                                    \\ \hline
\endfirsthead
%
\endhead
%
tOfDay.hours()           & Returns the hour component since midnight.              \\ \hline
tOfDay.minutes()         & Returns the minute component since midnight.            \\ \hline
tOfDay.seconds()         & Returns the second component since midnight.            \\ \hline
tOfDay.subseconds()      & Returns the fractional second component since midnight. \\ \hline
tOfDay.to\_duration()    & Returns the time duration since midnight.               \\ \hline
\begin{tabular}[c]{@{}l@{}}std::chrono::make12(hr)\\ std::chrono::make24(hr)\end{tabular} & Returns the 12-hour(24-hour) equivalent of a 24-hour(12-hour) format time. \\ \hline
\begin{tabular}[c]{@{}l@{}}std::chrono::is\_am(hr)\\ std::chrono::is\_pm(hr)\end{tabular} & Detects wether the 24-hour format time is a.m. or p.m. .                   \\ \hline
\end{longtable}

\mySamllsection{Calendar}

Calendar stands for various calendar dates such as year, a month, a weekday, or the n-th day of a week.

\filename{The current time}

\begin{cpp}
// currentTime.cpp
...
#include <chrono>
using std::chrono;
...
auto now = system_clock::now();
std::cout << "The current time is " << now << " UTC\n";

auto currentYear = year_month_day(floor<days>(now)).year();
std::cout << "The current year is " << currentYear << '\n';

auto h = floor<hours>(now) - sys_days(January/1/currentYear);
std::cout << "It has been " << h << " since New Years!\n";

std::cout << '\n';

auto birthOfChrist = year_month_weekday(sys_days(January/01/0000));
std::cout << "Weekday: " << birthOfChrist.weekday() << '\n';
\end{cpp}

The output of the program shows information referring to the current time.

\myGraphic{0.9}{content/chapter2/images/4.jpg}{Cyclic references}

The following table gives an overview of the calendar types.

\begin{center}
Various calendar types
\end{center}

% Please add the following required packages to your document preamble:
% \usepackage{longtable}
% Note: It may be necessary to compile the document several times to get a multi-page table to line up properly
\begin{longtable}[c]{|l|l|}
\hline
\textbf{Class}   & \textbf{Decription}                                     \\ \hline
\endfirsthead
%
\endhead
%
last\_spec       & Indicates the last day or weekday in a month.           \\ \hline
day              & Represents a day of a month.                            \\ \hline
month            & Represents a month of a year.                           \\ \hline
year             & Represents a year in the Gregorian calendar.            \\ \hline
weekday          & Represents a day of the week in the Gregorian calendar. \\ \hline
weekday\_indexed & Represents the n-th weekday of a month                  \\ \hline
weekday\_last    & Represents the last weekday of a month                  \\ \hline
month\_day       & Represents a specific day of a specific month.          \\ \hline
month\_day\_last & Represents the last day of a specific month.            \\ \hline
month\_weekday   & Represents the n-th weekday of a specific month.        \\ \hline
month\_weekday\_last       & Represents the last weekday of a specific month.          \\ \hline
year\_month      & Represents a specific month of a specific year.         \\ \hline
year\_month\_day & Represents a spefific year, month, and day.             \\ \hline
year\_month\_day\_last     & Represents the last day of a specific year and month      \\ \hline
year\_month\_weekday       & Represents the last day of a specific year and month      \\ \hline
year\_month\_weekday\_last & Represents the last weekday of a specific year and month. \\ \hline
\end{longtable}


\mySamllsection{Time Zone}

Time zones represent time specific to a geographic area. The following program snippet displays the local time in various time zones.

\filename{Local time is displayed in various time zones}

\begin{cpp}
// timezone.cpp
...
#include <chrono>
using std::chrono;
..
auto time = floor<milliseconds>(system_clock::now());
auto localTime = zoned_time<milliseconds>(current_zone(), time);
auto berlinTime = zoned_time<milliseconds>("Europe/Berlin", time);
auto newYorkTime = std::chorno::zoned_time<milliseconds>("America/New_York", time);
auto tokyoTime = zoned_time<milliseconds>("Asia/Tokyo", time);

std::cout << time << '\n';
std::cout << localTime << '\n';
std::cout << berlinTime << '\n';
std::cout << newYorkTime << '\n';
std::cout << tokyoTime << '\n';
\end{cpp}

The time zone functionality supports the access of the \href{https://www.iana.org/time-zones}{IANA time zone database}, enables the operation with various time zones, and provides information about leap seconds.

The following table gives an overview of the time zone functionality. For more detailed information, refer to \href{https://en.cppreference.com/w/cpp/chrono}{cpprefere.com}.

\begin{center}
Time zone information
\end{center}

% Please add the following required packages to your document preamble:
% \usepackage{longtable}
% Note: It may be necessary to compile the document several times to get a multi-page table to line up properly
\begin{longtable}[c]{|l|l|}
\hline
\textbf{Type} & \textbf{Description}                     \\ \hline
\endfirsthead
%
\endhead
%
tzdb          & Describes the IANA time zone database.   \\ \hline
locate\_zone  & Locates a time\_zone bases on its name.  \\ \hline
current\_zone & Returns the current time\_zone.          \\ \hline
time\_zone    & Represents a time zone.                  \\ \hline
sys\_info    & Returns information about a time zone at a specific time point.       \\ \hline
local\_info  & Represents information about the local time to UNIX time conventions. \\ \hline
zoned\_time   & Represents a time zone and a time point. \\ \hline
leap\_second & Contains information about a leap second insertion.                   \\ \hline
\end{longtable}

\mySamllsection{Chrono I/O}

The function std::chrono::parse parses a chrono object from a stream.

\filename{Parsing a time point and a time zone}

\begin{cpp}
std::istringstream inputStream{"1999-10-31 01:30:00 -08:00 US/Pacific"};

std::chrono::local_seconds timePoint;
std::string timeZone;
inputStream >> std::chrono::parse("%F %T %Ez %Z", timePoint, timeZone);
\end{cpp}

The parsing functionality provides various format specifiers to deal with the time of day and calendar dates such as year, month, week, and day. \href{https://en.cppreference.com/w/cpp/chrono/parse}{cppreference.com} provides detailed information to the format specifiers.





















