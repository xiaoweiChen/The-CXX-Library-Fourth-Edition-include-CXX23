std::formathas the following syntax: std::format(FormatString, Args) 

The format string FormatString consists of

\begin{itemize}
\item 
Ordinary characters (except \{ and \})

\item 
Escape sequences \{\{ and \}\} that are replaced by \{ and \}

\item 
Replacement fields
\end{itemize}

A replacement field has the format

\begin{itemize}
\item 
Beginning character \{
\begin{itemize}
\item 
Argument-ID (optional)

\item 
Colon : followed by a format specification (optional)
\end{itemize}

\item 
Closing character \}
\end{itemize}

The argument-id allows you to specify the index of the arguments in Args. The id’s start with 0. When you don’t provide the argument-id, the arguments are used as given. Either all replacement fields have to use an argument-id or none.

std::formatter and its specializations define the format specification for the arguments.

\begin{itemize}
\item 
basic types and string types: \href{https://en.cppreference.com/w/cpp/utility/format/formatter#Standard_format_specification}{standard format specification} based on \href{https://docs.python.org/3/library/string.html#formatspec}{Python’s format specification}.

\item 
chrono types: \href{https://en.cppreference.com/w/cpp/chrono/system_clock/formatter#Format_specification}{chrono format specification}

\item 
other types: user-defined format specification
\end{itemize}



























