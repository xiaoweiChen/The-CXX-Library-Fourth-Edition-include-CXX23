
要格式化用户定义类型,我必须为用户定义类型专门化\href{https://en.cppreference.com/w/cpp/utility/format/formatter}{std::formatter}类。所以,必须实现成员函数parse和format。

\begin{itemize}
\item 
parse:

\begin{itemize}
\item 
接受解析上下文。

\item 
解析解析后的上下文

\item 
返回一个迭代器到格式规范的末尾

\item 
出现错误时抛出std::format\_error
\end{itemize}

\item 
format:

\begin{itemize}
\item 
获取应进行格式化的值t和格式上下文fc

\item 
根据格式上下文进行格式化

\item 
将输出写入fc.out()

\item 
返回一个表示输出结束的迭代器
\end{itemize}

\end{itemize}

把理论付诸实践,格式化一个std::vector。

\mySamllsection{格式化一个std::vector}

类std::formatter的特化尽可能简单,指定了应用于容器的每个元素的格式规范。

\begin{cpp}
// formatVector.cpp

#include <format>

...

template <typename T>
struct std::formatter<std::vector<T>> {
	
	std::string formatString;
	
	auto constexpr parse(format_parse_context& ctx) { // (3)
		formatString = "{:";
			std::string parseContext(std::begin(ctx), std::end(ctx));
			formatString += parseContext;
			return std::end(ctx) - 1;
		}
	template <typename FormatContext>
	auto format(const std::vector<T>& v, FormatContext& ctx) {
		auto out= ctx.out();
		std::format_to(out, "[");
		if (v.size() > 0) std::format_to(out, formatString, v[0]);
		for (int i= 1; i < v.size(); ++i) {
			std::format_to(out, ", " + formatString, v[i]); // (1)
		}
		std::format_to(out, "]"); // (2)
		return std::format_to(out, "\n" );
	}
};

...

std::vector<int> myInts{1, 2, 3, 4, 5};
std::cout << std::format("{:}", myInts); // [1, 2, 3, 4, 5] // (4)
std::cout << std::format("{:+}", myInts); // [+1, +2, +3, +4, +5]
std::cout << std::format("{:03d}", myInts); // [001, 002, 003, 004, 005]
std::cout << std::format("{:b}", myInts); // [1, 10, 11, 100, 101] // (5)

std::vector<std::string> myStrings{"Only", "for", "testing"};
std::cout << std::format("{:}", myStrings); // [Only, for, testing]
std::cout << std::format("{:.3}", myStrings); // [Onl, for, tes]
\end{cpp}

std::vector的特化具有成员函数parse和format。parse实际上创建了一个formatString,应用于std::vector(1和2)的每个元素。解析上下文ctx(3)包含冒号(:)和右花括号(\})之间的字符,该函数返回一个指向右花括号(\})的迭代器。成员函数格式的工作更有趣,format上下文返回输出迭代器。使用了输出迭代器和\href{https://en.cppreference.com/w/cpp/utility/format/format_to}{std::format\_to}函数,所以std::vector的元素可以很好地显示出来。

std::vector的元素有几种格式化方式。第一行(4)显示数字。下面一行在每个数字前写一个符号,将它们对齐为3个字符,并使用0作为填充字符,第(5)行以二进制格式显示。剩下的两行输出std::vector的每个字符串,最后一行将每个字符串截断为三个字符。














































