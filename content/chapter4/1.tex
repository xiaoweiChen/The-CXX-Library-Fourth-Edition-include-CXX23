
\myGraphic{0.8}{content/chapter4/images/2.jpg}{}

std::array将C数组的内存和运行时特性与std::vector接口结合起来。

\href{http://en.cppreference.com/w/cpp/container/array}{std::array} 是固定长度的均匀容器,需要头文件<array>。所以std::array知道它的大小。

要初始化std::array,必须遵循一些特殊规则。\\

\noindent
\textbf{std::array<int, 10> arr}

这10个元素没有初始化。

\noindent
\textbf{std::array<int, 10> arr{}}

这10个元素默认初始化。

\noindent
\textbf{std::array<int, 10> arr\{1, 2, 3, 4, 5\}}

其余元素默认初始化。\\

std::array支持三种类型的索引访问。

\begin{cpp}
arr[n];
arr.at(n);
std::get<n>(arr);
\end{cpp}

最常用的带尖括号的第一类格式不检查arr的边界。这与art.at(n)相反。你最终会得到一个std::range-error异常。最后一个类型显示了std::array与std::tuple的关系,它们都是固定长度的容器。

这里有一些关于std::array的算术运算。

\filename{std::array}

\begin{cpp}
// array.cpp
...
#include <array>
...
std::array<int, 10> arr{1, 2, 3, 4, 5, 6, 7, 8, 9, 10};
for (auto a: arr) std::cout << a << " " ; // 1 2 3 4 5 6 7 8 9 10

double sum= std::accumulate(arr.begin(), arr.end(), 0);
std::cout << sum << '\n'; // 55

double mean= sum / arr.size();
std::cout << mean << '\n'; // 5.5
std::cout << (arr[0] == std::get<0>(arr)); // true
\end{cpp}




















