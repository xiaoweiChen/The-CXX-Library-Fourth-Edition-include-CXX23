
\myGraphic{0.8}{content/chapter4/images/4.jpg}{}

\href{http://en.cppreference.com/w/cpp/container/deque}{std::deque}, which typically consists of a sequence of fixed-sized arrays, is quite similar to std::vector. std::deque need the header <deque>. The std::deque has three additional member functions, deq.push\_front(elem), deq.pop\_front() and deq.emplace\_front(args... ) to add or remove elements at its beginning.

\filename{std::deque}

\begin{cpp}
// deque.cpp
...
#include <deque>
...
struct MyInt{
	MyInt(int i): myInt(i){};
	int myInt;
};

std::deque<MyInt> myIntDeq;

myIntDeq.push_back(MyInt(5));
myIntDeq.emplace_back(1);
std::cout << myIntDeq.size() << '\n'; // 2

std::deque<MyInt> intDeq;
intDeq.assign({1, 2, 3});
for (auto v: intDeq) std::cout << v << " "; // 1 2 3

intDeq.insert(intDeq.begin(), 0);
for (auto v: intDeq) std::cout << v << " "; // 0 1 2 3

intDeq.insert(intDeq.begin()+4, 4);
for (auto v: intDeq) std::cout << v << " "; // 0 1 2 3 4

intDeq.insert(intDeq.end(), {5, 6, 7, 8, 9, 10, 11});
for (auto v: intDeq) std::cout << v << " "; // 0 1 2 3 4 5 6 7 8 9 10 11

for (auto revIt= intDeq.rbegin(); revIt != intDeq.rend(); ++revIt)
	std::cout << *revIt << " "; // 11 10 9 8 7 6 5 4 3 2 1 0

intDeq.pop_back();
for (auto v: intDeq) std::cout << v << " "; // 0 1 2 3 4 5 6 7 8 9 10

intDeq.push_front(-1);
for (auto v: intDeq) std::cout << v << " "; // -1 0 1 2 3 4 5 6 7 8 9 10
\end{cpp}











































