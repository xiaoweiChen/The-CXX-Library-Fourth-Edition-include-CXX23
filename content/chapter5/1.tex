八个有序和无序容器都有一个共同点,其将键与值关联起来,可以使用键来获取值。要对关联容器进行分类,必须了解三个问题:

\begin{itemize}
\item 
键是否已排序?

\item 
键是否有关联的值?

\item 
一个键可以出现多次吗?
\end{itemize}

下表$2^3= 8$行给出了这三个问题的答案。我回答表格中的第四个问题。最好的情况下,键的访问时间有多快?


\begin{center}
关联容器的特征
\end{center}

% Please add the following required packages to your document preamble:
% \usepackage{longtable}
% Note: It may be necessary to compile the document several times to get a multi-page table to line up properly
\begin{longtable}[c]{|l|l|l|l|l|}
\hline
\textbf{关联容器} & \textbf{排序} & \textbf{关联值} & \textbf{相同的键} & \textbf{访问时间} \\ \hline
\endfirsthead
%
\endhead
%
std::set                 & yes & no  & no  & 对数 \\ \hline
std::unordered\_set      & no  & no  & no  & 常数    \\ \hline
std::map                 & yes & yes & no  & 对数 \\ \hline
std::unordered\_map      & no  & yes & no  & 常数    \\ \hline
std::multiset            & yes & no  & yes & 对数 \\ \hline
std::unordered\_multiset & no  & no  & yes & 常数    \\ \hline
std::multimap            & yes & yes & yes & 对数 \\ \hline
std::unordered\_multimap & no  & yes & yes & 常数    \\ \hline
\end{longtable}

自C++98起,就有了顺序关联容器,C++11还增加了无序关联容器。这两个类都有非常相似的接口。这就是为什么下面的代码示例对于std::map和std::unordered\_map是相同的。更准确地说,std::unordered\_map的接口是std::map接口的超集。其余三个无序关联容器也是如此,将代码从有序容器移植到无序容器很容易。

可以使用初始化列表初始化容器,并使用索引操作符添加新元素。要访问键/值对p的第一个元素,首先使用p,对于第二个元素,使用p.second。p.first是键,p.second是对子的关联值。

\filename{std::map 与 std::unordered\_map}

\begin{cpp}
// orderedUnorderedComparison.cpp
...
#include <map>
#include <unordered_map>

// std::map

std::map<std::string, int> m {{"Dijkstra", 1972}, {"Scott", 1976}};
m["Ritchie"]= 1983;
std::cout << m["Ritchie"]; // 1983
for(auto p : m) std::cout << "{" << p.first << "," << p.second << "}";
							// {Dijkstra,1972},{Ritchie,1983},{Scott,1976}
m.erase("Scott");
for(auto p : m) std::cout << "{" << p.first << "," << p.second << "}";
								// {Dijkstra,1972},{Ritchie,1983}
m.clear();
std::cout << m.size() << '\n'; // 0

// std::unordered_map
std::unordered_map<std::string, int> um {{"Dijkstra", 1972}, {"Scott", 1976}};
um["Ritchie"]= 1983;
std::cout << um["Ritchie"]; // 1983
for(auto p : um) std::cout << "{" << p.first << "," << p.second << "}";
								// {Ritchie,1983},{Scott,1976},{Dijkstra,1972}
um.erase("Scott");
for(auto p : um) std::cout << "{" << p.first << "," << p.second << "}";
								// {Ritchie,1983},{Dijkstra,1972}

um.clear();
std::cout << um.size() << '\n'; // 0
\end{cpp}

这两个程序执行之间有一个微妙的区别:std::map的键是有序的,而std::unordered\_map的键是无序的。问题是:为什么在C++中有这样类似的容器?我已经在表格中指出,对于C++来说,答案通常是:性能。对无序关联容器的键的访问时间是常量,因此与容器的大小无关。若容器足够大,性能差异是显著的。

\mySamllsection{Contains}

成员函数associativeContainer.contains(ele)检查associativeContainer是否有元素ele。

\filename{检查关联容器是否有元素}

\begin{cpp}
// containsElement.cpp
...
template <typename AssozCont>
bool containsElement5(const AssozCont& assozCont) {
	return assozCont.contains(5);
}

std::set<int> mySet{1, 2, 3, 4, 5, 6, 7, 8, 9, 10};
std::cout << "containsElement5(mySet): "
		  << containsElement5(mySet); // true

std::unordered_set<int> myUnordSet{1, 2, 3, 4, 5, 6, 7, 8, 9, 10};
std::cout << "containsElement5(myUnordSet): "
		  << containsElement5(myUnordSet); // true

std::map<int, std::string> myMap{ {1, "red"}, {2, "blue"}, {3, "green"} };
std::cout << "containsElement5(myMap): "
		  << containsElement5(myMap); // false

std::unordered_map<int, std::string> myUnordMap{ {1, "red"}, {2, "blue"},
												 {3, "green"} };
std::cout << "containsElement5(myUnordMap): "
		  << containsElement5(myUnordMap); // false
\end{cpp}


\mySamllsection{插入和删除}

关联容器中元素的插入(插入和放置)和删除(擦除)操作类似于std::vector的规则。对于一个键只能有一次的关联容器,若键已经在容器中,则插入失败。此外,有序关联容器支持一个特殊的函数ordAssCont.erase(key),会删除所有具有键的对子,并返回其数量。

\filename{插入和删除}

\begin{cpp}
// associativeContainerModify.cpp
...
#include <set>
...
std::multiset<int> mySet{3, 1, 5, 3, 4, 5, 1, 4, 4, 3, 2, 2, 7, 6, 4, 3, 6};

for (auto s: mySet) std::cout << s << " ";
	// 1 1 2 2 3 3 3 3 4 4 4 4 5 5 6 6 7

mySet.insert(8);
std::array<int, 5> myArr{10, 11, 12, 13, 14};
mySet.insert(myArr.begin(), myArr.begin()+3);
mySet.insert({22, 21, 20});
for (auto s: mySet) std::cout << s << " ";
	// 1 1 2 2 3 3 3 3 4 4 4 4 5 5 6 6 7 10 11 12 20 21 22
	
std::cout << mySet.erase(4); // 4
mySet.erase(mySet.lower_bound(5), mySet.upper_bound(15));
for (auto s: mySet) std::cout << s << " ";
	// 1 1 2 2 3 3 3 3 20 21 22
\end{cpp}


