\myGraphic{0.6}{content/chapter9/images/1.jpg}{Cippi slides down the slide}


\begin{myNotic}{This chapter is intentionally not exhaustive}
This book is about the C++ Standard library. Therefore, I will not go into the details of callable units. I provide as much information as necessary to use them in the Standard Template Library algorithms correctly. An exhaustive discussion of callable units should be part of a book about the C++ core language.
\end{myNotic}

Many STL algorithms and containers can be parametrized with callable units (short callable). A callable is something that behaves like a function. Not only are these functions but also function objects and lambda functions. Predicates are special functions that return a boolean as a result. If a predicate has one argument, it’s called a unary predicate. If a predicate has two arguments, it’s called a binary predicate. The same holds for functions. A function taking one argument is a unary function; a function taking two arguments is a binary function.

\begin{myTip}{To change the elements of a container, your algorithm should take them by reference}
	
Callables can receive their arguments by value or reference from their container. To modify the container elements, they must address them directly, so the callable must get them by reference.
\end{myTip}