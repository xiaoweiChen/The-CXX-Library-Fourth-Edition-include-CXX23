
The stream classes std::istream and std::ostream are often used for the reading and writing of data. Use of std::istream classes requires the <istream> header; use of std::ostream classes requires the <ostream> header. You can have both with the header <iostream>. std::istream is a typedef for the class basic\_istream and the character type char, std::ostream for the class basic\_ostream respectively:

\begin{cpp}
typedef basic_istream<char> istream;
typedef basic_ostream<char> ostream;
\end{cpp}

C++ has four predefined stream objects for the convenience of dealing with the keyboard and the monitor.

\begin{center}
The four predefined stream objects
\end{center}

% Please add the following required packages to your document preamble:
% \usepackage{longtable}
% Note: It may be necessary to compile the document several times to get a multi-page table to line up properly
\begin{longtable}[c]{|l|l|l|l|}
\hline
\textbf{Stream object} & \textbf{C pendant} & \textbf{Device} & \textbf{Buffered} \\ \hline
\endfirsthead
%
\endhead
%
std::cin               & stdin              & keyboard        & yes               \\ \hline
std::cout              & stdout             & console         & yes               \\ \hline
std::cerr              & stderr             & console         & no                \\ \hline
std::clog              &                    & monitor         & yes               \\ \hline
\end{longtable}


\begin{myNotic}{The stream objects are also available for wchar\_t}
The four stream objects for wchar\_t, std::wcin, std::wcout, std::wcerr, and std::wclog are by far not so heavily used as their char pendants. Therefore I treat them only marginally.
\end{myNotic}

The stream objects are sufficient to write a program that reads from the command line and returns the sum.

\filename{The stream objects}

\begin{cpp}
// IOStreams.cpp
...
#include <iostream>

int main(){
	std::cout << "Type in your numbers";
	std::cout << "(Quit with an arbitrary character): " << std::endl;
				// 2000 <Enter> 11 <a>
	int sum{0};
	int val;
	while (std::cin >> val) sum += val;
	std::cout << "Sum: " << sum; // Sum: 2011
}
\end{cpp}

The minor program above uses the stream operators <{}< and >{}> and the stream manipulator std::endl.

The insert operator <{}< pushes characters onto the output stream std::cout; the extract operator >{}> pulls the characters from the input stream std::cin. You can build chains of insert or extract operators because both operators return a reference to themselves.

std::endl is a stream manipulator because it puts a ‘\verb|\|n’ character onto std::cout and flushes the output buffer.

Here are the most frequently used stream manipulators.

\begin{center}
The most frequently used stream manipulators
\end{center}

% Please add the following required packages to your document preamble:
% \usepackage{longtable}
% Note: It may be necessary to compile the document several times to get a multi-page table to line up properly
\begin{longtable}[c]{|l|l|l|}
\hline
\textbf{Manipulator} & \textbf{Stream type} & \textbf{Description}                                 \\ \hline
\endfirsthead
%
\endhead
%
std::endl            & output               & Inserts a new-line character and flushes the stream. \\ \hline
std::flush           & output               & Flushes the stream.                                  \\ \hline
std::ws              & input                & Discards leading whitespace.                         \\ \hline
\end{longtable}

\mySamllsection{Input}

You can read in C++ in two ways from the input stream: Formatted with the extractor >> and unformatted with explicit member functions.

\mySamllsection{Formatted Input}

The extraction operator >{}>

\begin{itemize}
\item 
is predefined for all built-in types and strings,

\item 
can be implemented for user-defined data types,

\item 
can be configured by format specifiers.
\end{itemize}

\begin{myTip}{std::cin ignores by default leading whitespace}
\begin{cpp}
#include <iostream>
...
int a, b;
std::cout << "Two natural numbers: " << '\n';
std::cin >> a >> b; // < 2000 11>
std::cout << "a: " << a << " b: " << b;
\end{cpp}
\end{myTip}

\mySamllsection{Unformatted Input}

There are many member functions for the unformatted input from an input stream is.

\begin{center}
Unformatted input from an input stream
\end{center}

% Please add the following required packages to your document preamble:
% \usepackage{longtable}
% Note: It may be necessary to compile the document several times to get a multi-page table to line up properly
\begin{longtable}[c]{|l|l|}
\hline
\textbf{Member Function}                          & \textbf{Description}                                                                 \\ \hline
\endfirsthead
%
\endhead
%
is.get(ch)                                        & Reads one character into ch.                                                         \\ \hline
is.get(buf, num)                                  & Reads at most num characters into the buffer buf.                                    \\ \hline
is.getline(buf, num{[}, delim{]}) &
\begin{tabular}[c]{@{}l@{}}Reads at most num characters into the buffer buf.\\ Uses the line-delimiter delim(default \textbackslash{}n) optionally.\end{tabular} \\ \hline
is.gcount()                                       & Returns the number of last extracted characters from is by an unformatted operation. \\ \hline
is.ignore(streamsize sz=1, int delim=end-of-file) & Ignores sz characters until delim.                                                   \\ \hline
is.peek()                                         & Gets one character from is without consuming it.                                     \\ \hline
is.unget()                                        & Pushes the last read character back to is.                                           \\ \hline
is.putback(ch)                                    & Pushes the character ch onto the stream is.                                          \\ \hline
\end{longtable}

\begin{myTip}{std::string has a getline function}
The getline function of std::string has a big advantage above the getline function of the istream. The std::string automatically takes care of its memory. On the contrary, you must reserve the memory for the buffer buf in the is.get(buf, num) function.
\end{myTip}

\begin{cpp}
// inputUnformatted.cpp
...
#include <iostream>
...
std::string line;
std::cout << "Write a line: " << '\n';

std::getline(std::cin, line); // <Only for testing purpose.>
std::cout << line << '\n'; // Only for testing purpose.

std::cout << "Write numbers, separated by;" << '\n';
while (std::getline(std::cin, line, ';') ) {
	std::cout << line << " ";
} // <2000;11;a>
  // 2000 11
\end{cpp}


\mySamllsection{Output}

You can push characters with the insert operator <{}< onto the output stream.

The insert operator <{}<

\begin{itemize}
\item 
is predefined for all built-in types and strings,

\item 
can be implemented for user-defined data types,

\item 
can be adjusted by format specifiers.
\end{itemize}


\mySamllsection{Format Specifier}

Format specifiers enable you to adjust the input and output data explicitly.

\begin{myNotic}{I use manipulators as format specifiers}
The format specifiers are available as manipulators and flags. I only present manipulators in this book because their functionality is quite similar, and manipulators are more comfortable to use.

\filename{Manipulators as format specifiers}

\begin{cpp}
// formatSpecifier.cpp
...
#include <iostream>
...
int num{2011};

std::cout.setf(std::ios::hex, std::ios::basefield);
std::cout << num << '\n'; // 7db
std::cout.setf(std::ios::dec, std::ios::basefield);
std::cout << num << '\n'; // 2011

std::cout << std::hex << num << '\n'; // 7db
std::cout << std::dec << num << '\n'; // 2011
\end{cpp}

\end{myNotic}

The following tables present the important format specifiers. The format specifiers are sticky except for the field width, which is reset after each application.

The manipulators without arguments require the header <iostream>; the manipulators with arguments require the header <iomanip>.

\begin{center}
Displaying of boolean values
\end{center}

% Please add the following required packages to your document preamble:
% \usepackage{longtable}
% Note: It may be necessary to compile the document several times to get a multi-page table to line up properly
\begin{longtable}[c]{|l|l|l|}
\hline
\textbf{Manipulator} & \textbf{Stream type} & \textbf{Description}                       \\ \hline
\endfirsthead
%
\endhead
%
std::boolalpha       & input and output     & Displays the boolean as a word.            \\ \hline
std::noboolalpha     & input and output     & Displays the boolean as a number(default). \\ \hline
\end{longtable}

\begin{center}
Set the field width and the fill character
\end{center}

% Please add the following required packages to your document preamble:
% \usepackage{longtable}
% Note: It may be necessary to compile the document several times to get a multi-page table to line up properly
\begin{longtable}[c]{|l|l|l|}
\hline
\textbf{Manipulator} & \textbf{Stream type} & \textbf{Description}                            \\ \hline
\endfirsthead
%
\endhead
%
std::setw(val)       & input and output     & Sets the field width to val.                    \\ \hline
std::setfill(c)      & output stream        & Sets the fill character to c (default: spaces). \\ \hline
\end{longtable}



\begin{center}
Alignment of the text
\end{center}

% Please add the following required packages to your document preamble:
% \usepackage{longtable}
% Note: It may be necessary to compile the document several times to get a multi-page table to line up properly
\begin{longtable}[c]{|l|l|l|}
\hline
\textbf{Manipulator} & \textbf{Stream type} & \textbf{Description}                                \\ \hline
\endfirsthead
%
\endhead
%
std::left            & output               & Aligns the output left.                             \\ \hline
std::right           & output               & Aligns the output right.                            \\ \hline
std::internal        & output               & Aligns the signs of numbers left, the values right. \\ \hline
\end{longtable}

\begin{center}
Positive signs and upper or lower case
\end{center}


% Please add the following required packages to your document preamble:
% \usepackage{longtable}
% Note: It may be necessary to compile the document several times to get a multi-page table to line up properly
\begin{longtable}[c]{|l|l|l|}
\hline
\textbf{Manipulator} & \textbf{Stream type} & \textbf{Description}                                    \\ \hline
\endfirsthead
%
\endhead
%
std::showpos         & output               & Displays positive signs.                                \\ \hline
std::noshowpos       & output               & Doesn't display positive signs(default).                \\ \hline
std::uppercase       & output               & Uses upper case characters for numeric values(defulat). \\ \hline
std::lowercase       & output               & Uses lower case characters for numeric values.          \\ \hline
\end{longtable}


\begin{center}
Display of the numeric base
\end{center}

% Please add the following required packages to your document preamble:
% \usepackage{longtable}
% Note: It may be necessary to compile the document several times to get a multi-page table to line up properly
\begin{longtable}[c]{|l|l|l|}
\hline
\textbf{Manipulator} & \textbf{Stream type} & \textbf{Description}                             \\ \hline
\endfirsthead
%
\endhead
%
std::oct             & input and output     & Uses natural numbers in octal format.            \\ \hline
std::dec             & input and output     & Uses natural numbers in decimal format(default). \\ \hline
std::hex             & input and output     & Uses natural numbers in hexadecimal format.      \\ \hline
std::showbase        & output               & Displays the numeric base.                       \\ \hline
std::noshowbase      & output               & Doesn't display the numeric base(default).       \\ \hline
\end{longtable}

There are special rules for floating-point numbers:

\begin{itemize}
\item 
The number of significant digits (digits after the comma) is, by default, six.

\item 
If the number of significant digits is not big enough, the number is displayed in scientific notation.
 
\item 
Leading and trailing zeros are not displayed.

\item 
If possible, the decimal point is not displayed.
\end{itemize}


\begin{center}
Floating point numbers
\end{center}

% Please add the following required packages to your document preamble:
% \usepackage{longtable}
% Note: It may be necessary to compile the document several times to get a multi-page table to line up properly
\begin{longtable}[c]{|l|l|l|}
\hline
\textbf{Manipulator}   & \textbf{Stream type} & \textbf{Description}                                      \\ \hline
\endfirsthead
%
\endhead
%
std::setprecision(val) & output               & Adjusts the precision of the output to val.               \\ \hline
std::showpoint         & output               & Displays the decimal point.                               \\ \hline
std::noshowpoint       & output               & Doesn't display the decimal point(default).               \\ \hline
std::fixed             & output               & Displays the floating-point number in decimal format.     \\ \hline
std::scientific        & output               & Displays the floating-point number in scientific format.  \\ \hline
std::hecfloat          & output               & Displays the floating-point number in hexadecimal format. \\ \hline
std::defaultfloat & output & Displays the floating-point number in default floating-point notation. \\ \hline
\end{longtable}

\filename{Format specifier}

\begin{cpp}
// formatSpecifierOutput.cpp
...
#include <iomanip>
#include <iostream>
...

std::cout.fill('#');
std::cout << -12345;
std::cout << std::setw(10) << -12345; // ####-12345
std::cout << std::setw(10) << std::left << -12345; // -12345####
std::cout << std::setw(10) << std::right << -12345; // ####-12345
std::cout << std::setw(10) << std::internal << -12345; //-####12345

std::cout << std::oct << 2011; // 3733
std::cout << std::hex << 2011; // 7db

std::cout << std::showbase;
std::cout << std::dec << 2011; // 2011
std::cout << std::oct << 2011; // 03733
std::cout << std::hex << 2011; // 0x7db

std::cout << 123.456789; // 123.457
std::cout << std::fixed;
std::cout << std::setprecision(3) << 123.456789; // 123.457
std::cout << std::setprecision(6) << 123.456789; // 123.456789
std::cout << std::setprecision(9) << 123.456789; // 123.456789000

std::cout << std::scientific;
std::cout << std::setprecision(3) << 123.456789; // 1.235e+02
std::cout << std::setprecision(6) << 123.456789; // 1.234568e+02
std::cout << std::setprecision(9) << 123.456789; // 1.234567890e+02

std::cout << std::hexfloat;
std::cout << std::setprecision(3) << 123.456789; // 0x1.edd3c07ee0b0bp+6
std::cout << std::setprecision(6) << 123.456789; // 0x1.edd3c07ee0b0bp+6
std::cout << std::setprecision(9) << 123.456789; // 0x1.edd3c07ee0b0bp+6

std::cout << std::defaultfloat;
std::cout << std::setprecision(3) << 123.456789; // 123
std::cout << std::setprecision(6) << 123.456789; // 123.457
std::cout << std::setprecision(9) << 123.456789; // 123.456789

\end{cpp}








