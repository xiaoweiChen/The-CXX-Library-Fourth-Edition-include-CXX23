
流类std::istream和std::ostream通常用于数据的读写。使用std::istream类需要<istream>头文件;使用std::ostream类需要<ostream>头文件,可以同时使用<iostream>头文件。std::istream分别是类basic\_istream和类basic\_ostream的字符类型char, std::ostream的类型定义:

\begin{cpp}
typedef basic_istream<char> istream;
typedef basic_ostream<char> ostream;
\end{cpp}

为了方便处理键盘和显示器,C++有四个预定义的流对象。

\begin{center}
四个预定义的流对象
\end{center}

% Please add the following required packages to your document preamble:
% \usepackage{longtable}
% Note: It may be necessary to compile the document several times to get a multi-page table to line up properly
\begin{longtable}[c]{|l|l|l|l|}
\hline
\textbf{流对象} & \textbf{C} & \textbf{设备} & \textbf{是否缓冲} \\ \hline
\endfirsthead
%
\endhead
%
std::cin               & stdin              & keyboard        & yes               \\ \hline
std::cout              & stdout             & console         & yes               \\ \hline
std::cerr              & stderr             & console         & no                \\ \hline
std::clog              &                    & monitor         & yes               \\ \hline
\end{longtable}


\begin{myNotic}{流对象也可用于wchar\_t}
到目前为止,wchar\_t、std::wcin、std::wcout、std::wcerr和std::wclog的四个流对象并没有像它们的字符那样大量使用。这里,只是稍微提一下。
\end{myNotic}

流对象足以编写从命令行读取并返回总和的程序。

\filename{流对象}

\begin{cpp}
// IOStreams.cpp
...
#include <iostream>

int main(){
	std::cout << "Type in your numbers";
	std::cout << "(Quit with an arbitrary character): " << std::endl;
				// 2000 <Enter> 11 <a>
	int sum{0};
	int val;
	while (std::cin >> val) sum += val;
	std::cout << "Sum: " << sum; // Sum: 2011
}
\end{cpp}

上面的小程序使用了流操作符<{}<和>{}>,以及流操纵符std::endl。

插入操作符<{}<将字符压入输出流std::cout;提取操作符>{}>从输入流std::cin中提取字符。可以构建插入或提取操作符链,这两个操作符都返回对自己的引用。

std::endl是一个流操纵符,因为它将一个'\verb|\|n'字符放到std::cout上,并刷新输出缓冲区。

下面是最常用的流操纵符。

\begin{center}
最常用的流操纵符
\end{center}

% Please add the following required packages to your document preamble:
% \usepackage{longtable}
% Note: It may be necessary to compile the document several times to get a multi-page table to line up properly
\begin{longtable}[c]{|l|l|l|}
\hline
\textbf{操纵符} & \textbf{流类型} & \textbf{描述}                                 \\ \hline
\endfirsthead
%
\endhead
%
std::endl            & output               & 插入新行字符并刷新流。 \\ \hline
std::flush           & output               & 刷新流。                                  \\ \hline
std::ws              & input                & 丢弃前置空格。                         \\ \hline
\end{longtable}

\mySamllsection{输入}

C++中,从输入流中读取数据有两种方式:用提取器>{}>格式化和用显式成员函数不格式化。

\mySamllsection{格式化输入}

提取操作符>{}>

\begin{itemize}
\item 
为所有内置类型和字符串预定义,

\item 
可为用户定义的数据类型进行实现,

\item 
可以通过格式说明符进行配置。
\end{itemize}

\begin{myTip}{默认情况下,std::cin会忽略前置空格}
\begin{cpp}
#include <iostream>
...
int a, b;
std::cout << "Two natural numbers: " << '\n';
std::cin >> a >> b; // < 2000 11>
std::cout << "a: " << a << " b: " << b;
\end{cpp}
\end{myTip}

\mySamllsection{非格式化输入}

对于来自输入流的未格式化输入,有许多成员函数。

\begin{center}
来自输入流的未格式化输入
\end{center}

% Please add the following required packages to your document preamble:
% \usepackage{longtable}
% Note: It may be necessary to compile the document several times to get a multi-page table to line up properly
\begin{longtable}[c]{|l|l|}
\hline
\textbf{成员函数}                          & \textbf{描述}                                                                 \\ \hline
\endfirsthead
%
\endhead
%
is.get(ch)                                        & 将一个字符读入ch。                                                         \\ \hline
is.get(buf, num)                                  & 在缓冲区中最多读取num个字符。                                    \\ \hline
is.getline(buf, num{[}, delim{]}) &
\begin{tabular}[c]{@{}l@{}}在缓冲区中最多读取num个字符。\\ 可选地使用行分隔符delim(默认\textbackslash{}n)。\end{tabular} \\ \hline
is.gcount()                                       & 返回通过未格式化操作从数组中提取的最后字符数。 \\ \hline
is.ignore(streamsize sz=1, int delim=end-of-file) & 忽略sz字符直到delim。                                                   \\ \hline
is.peek()                                         & 从is中获取一个字符,但不使用它。                                     \\ \hline
is.unget()                                        & 将最后一个读取的字符回推到is。                                           \\ \hline
is.putback(ch)                                    & 将字符ch引入流is。                                          \\ \hline
\end{longtable}

\begin{myTip}{std::string有一个getline函数}
std::string的getline函数比istream的getline函数有一个很大的优势,字符串会自动处理它的内存,但必须为is.get(buf, num)所使用的缓冲区保留内存。
\end{myTip}

\begin{cpp}
// inputUnformatted.cpp
...
#include <iostream>
...
std::string line;
std::cout << "Write a line: " << '\n';

std::getline(std::cin, line); // <Only for testing purpose.>
std::cout << line << '\n'; // Only for testing purpose.

std::cout << "Write numbers, separated by;" << '\n';
while (std::getline(std::cin, line, ';') ) {
	std::cout << line << " ";
} // <2000;11;a>
  // 2000 11
\end{cpp}


\mySamllsection{输出}

可以使用插入操作符<{}<将字符推入输出流。

插入操作符<{}<

\begin{itemize}
\item 
为所有内置类型和字符串预定义,

\item 
可以为用户定义的数据类型实现,

\item 
可以通过格式说明符进行调整。
\end{itemize}


\mySamllsection{格式说明符}

格式说明符能够显式地调整输入和输出数据。

\begin{myNotic}{操纵符作为格式说明符}
格式说明符可用作操纵符和标志。本书中只介绍操纵符,因为它们的功能非常相似,而且操纵符使用起来更顺手。

\filename{操纵符作为格式说明符}

\begin{cpp}
// formatSpecifier.cpp
...
#include <iostream>
...
int num{2011};

std::cout.setf(std::ios::hex, std::ios::basefield);
std::cout << num << '\n'; // 7db
std::cout.setf(std::ios::dec, std::ios::basefield);
std::cout << num << '\n'; // 2011

std::cout << std::hex << num << '\n'; // 7db
std::cout << std::dec << num << '\n'; // 2011
\end{cpp}

\end{myNotic}

下表给出了重要的格式说明符。除了字段宽度之外,格式说明符是固定的,在每个程序结束后重置。

没有实参的操作符需要头文件<iostream>,带参数的操作符需要头文件<iomanip>。

\begin{center}
布尔值的显示
\end{center}

% Please add the following required packages to your document preamble:
% \usepackage{longtable}
% Note: It may be necessary to compile the document several times to get a multi-page table to line up properly
\begin{longtable}[c]{|l|l|l|}
\hline
\textbf{操纵符} & \textbf{流类型} & \textbf{描述}                       \\ \hline
\endfirsthead
%
\endhead
%
std::boolalpha       & 输入和输出     & 将布尔值显示为字符。            \\ \hline
std::noboolalpha     & 输入和输出     & 将布尔值显示为数字(默认)。 \\ \hline
\end{longtable}

\begin{center}
设置字段宽度和填充字符
\end{center}

% Please add the following required packages to your document preamble:
% \usepackage{longtable}
% Note: It may be necessary to compile the document several times to get a multi-page table to line up properly
\begin{longtable}[c]{|l|l|l|}
\hline
\textbf{操纵符} & \textbf{流类型} & \textbf{描述}                            \\ \hline
\endfirsthead
%
\endhead
%
std::setw(val)       & 输入和输出     & 将字段宽度设置为val。                    \\ \hline
std::setfill(c)      & 输出流        & 将填充字符设置为c字符(默认为空格)。 \\ \hline
\end{longtable}

\begin{center}
文本对齐
\end{center}

% Please add the following required packages to your document preamble:
% \usepackage{longtable}
% Note: It may be necessary to compile the document several times to get a multi-page table to line up properly
\begin{longtable}[c]{|l|l|l|}
\hline
\textbf{操纵符} & \textbf{流类型} & \textbf{描述}                                \\ \hline
\endfirsthead
%
\endhead
%
std::left            & 输出               & 向左对齐输出。                             \\ \hline
std::right           & 输出               & 向右对齐输出。                            \\ \hline
std::internal        & 输出               & 左对齐数字符号,右对齐数值符号。 \\ \hline
\end{longtable}

\begin{center}
正号和大写或小写
\end{center}

% Please add the following required packages to your document preamble:
% \usepackage{longtable}
% Note: It may be necessary to compile the document several times to get a multi-page table to line up properly
\begin{longtable}[c]{|l|l|l|}
\hline
\textbf{操纵符} & \textbf{流类型} & \textbf{描述}                                    \\ \hline
\endfirsthead
%
\endhead
%
std::showpos         & 输出               & 显示正号。                                \\ \hline
std::noshowpos       & 输出               & 不显示正号(默认)。                \\ \hline
std::uppercase       & 输出               & 对数值使用大写字符(默认值)。 \\ \hline
std::lowercase       & 输出               & 对数值使用小写字符。          \\ \hline
\end{longtable}


\begin{center}
显示数字基数
\end{center}

% Please add the following required packages to your document preamble:
% \usepackage{longtable}
% Note: It may be necessary to compile the document several times to get a multi-page table to line up properly
\begin{longtable}[c]{|l|l|l|}
\hline
\textbf{操纵符} & \textbf{流类型} & \textbf{描述}                             \\ \hline
\endfirsthead
%
\endhead
%
std::oct             & 输出和输出     & 使用八进制格式的自然数。            \\ \hline
std::dec             & 输出和输出     & 使用十进制格式的自然数(默认)。 \\ \hline
std::hex             & 输出和输出     & 使用十六进制格式的自然数。      \\ \hline
std::showbase        & 输出               & 显示数字基数。                       \\ \hline
std::noshowbase      & 输出               & 不显示数字基数(默认)。       \\ \hline
\end{longtable}

对于浮点数有一些特殊的规则:

\begin{itemize}
\item 
默认情况下,有效数字(小数点后面的数字)的个数为6。

\item 
若有效位数不够大,则以科学记数法显示。
 
\item 
不显示前导零和尾随零。

\item 
若可能,则不显示小数点。
\end{itemize}


\begin{center}
浮点数
\end{center}

% Please add the following required packages to your document preamble:
% \usepackage{longtable}
% Note: It may be necessary to compile the document several times to get a multi-page table to line up properly
\begin{longtable}[c]{|l|l|l|}
\hline
\textbf{操纵符}   & \textbf{流类型} & \textbf{描述}                                      \\ \hline
\endfirsthead
%
\endhead
%
std::setprecision(val) & 输出               & 将输出精度调整为val。               \\ \hline
std::showpoint         & 输出               & 显示小数点。                               \\ \hline
std::noshowpoint       & 输出               & 不显示小数点(默认)。               \\ \hline
std::fixed             & 输出               & 以十进制格式显示浮点数。     \\ \hline
std::scientific        & 输出               & 以科学计数法显示浮点数。  \\ \hline
std::hecfloat          & 输出               & 以十六进制格式显示浮点数。 \\ \hline
std::defaultfloat & 输出 & 以默认浮点表示法显示浮点数。 \\ \hline
\end{longtable}

\filename{格式说明符}

\begin{cpp}
// formatSpecifierOutput.cpp
...
#include <iomanip>
#include <iostream>
...

std::cout.fill('#');
std::cout << -12345;
std::cout << std::setw(10) << -12345; // ####-12345
std::cout << std::setw(10) << std::left << -12345; // -12345####
std::cout << std::setw(10) << std::right << -12345; // ####-12345
std::cout << std::setw(10) << std::internal << -12345; //-####12345

std::cout << std::oct << 2011; // 3733
std::cout << std::hex << 2011; // 7db

std::cout << std::showbase;
std::cout << std::dec << 2011; // 2011
std::cout << std::oct << 2011; // 03733
std::cout << std::hex << 2011; // 0x7db

std::cout << 123.456789; // 123.457
std::cout << std::fixed;
std::cout << std::setprecision(3) << 123.456789; // 123.457
std::cout << std::setprecision(6) << 123.456789; // 123.456789
std::cout << std::setprecision(9) << 123.456789; // 123.456789000

std::cout << std::scientific;
std::cout << std::setprecision(3) << 123.456789; // 1.235e+02
std::cout << std::setprecision(6) << 123.456789; // 1.234568e+02
std::cout << std::setprecision(9) << 123.456789; // 1.234567890e+02

std::cout << std::hexfloat;
std::cout << std::setprecision(3) << 123.456789; // 0x1.edd3c07ee0b0bp+6
std::cout << std::setprecision(6) << 123.456789; // 0x1.edd3c07ee0b0bp+6
std::cout << std::setprecision(9) << 123.456789; // 0x1.edd3c07ee0b0bp+6

std::cout << std::defaultfloat;
std::cout << std::setprecision(3) << 123.456789; // 123
std::cout << std::setprecision(6) << 123.456789; // 123.457
std::cout << std::setprecision(9) << 123.456789; // 123.456789

\end{cpp}








