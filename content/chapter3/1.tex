每个容器都会提供不同的构造函数。要删除容器cont中所有的元素,可以使用cont.clear()。创建容器、删除容器、添加或删除元素都一样,每次容器都会自行负责内存的管理。

下表显示了容器的构造函数和析构函数。下面的表中,使用std:vector作为代表。

\begin{center}
创建和删除容器
\end{center}

% Please add the following required packages to your document preamble:
% \usepackage{longtable}
% Note: It may be necessary to compile the document several times to get a multi-page table to line up properly
\begin{longtable}[c]{|l|l|}
\hline
\textbf{类型}   & \textbf{示例}                                            \\ \hline
\endfirsthead
%
\endhead
%
默认构造         & std::vector\textless{}int\textgreater vec1                  \\ \hline
范围构造         & std::vector\textless{}int\textgreater vec2(vec1.begin(), vec1.end()) \\ \hline
复制构造            & std::vector\textless{}int\textgreater vec3(vec2)            \\ \hline
复制构造            & std::vector\textless{}int\textgreater vec3=vec2             \\ \hline
移动构造            & std::vector\textless{}int\textgreater vec4(std::move(vec3)) \\ \hline
移动构造            & std::vector\textless{}int\textgreater vec4=std::move(vec3)  \\ \hline
序列(初始化器列表) & std::vector\textless{}int\textgreater vec5\{1,2,3,4,5\}              \\ \hline
序列(初始化器列表) & std::vector\textless{}int\textgreater vec5=\{1,2,3,4,5\}             \\ \hline
析构      & vec5.$\sim$vector()                                         \\ \hline
删除所有元素 & vec5.clear()                                                \\ \hline
\end{longtable}

必须在编译时指定std::array的大小,并使用\href{https://en.cppreference.com/w/cpp/language/aggregate_initialization}{聚合初始化}进行初始化,std::array没有用于删除其元素的成员函数。

下面的例子中,我对不同的容器使用了不同的构造函数。

\filename{不同的构造函数}

\begin{cpp}
// containerConstructor.cpp
...
#include <map>
#include <unordered_map>
#include <vector>
...
using namespace std;

vector<int> vec= {1, 2, 3, 4, 5, 6, 7, 8, 9};
map<string, int> m= {{"bart", 12345}, {"jenne", 34929}, {"huber", 840284} };
unordered_map<string, int> um{m.begin(), m.end()};

for (auto v: vec) cout << v << " "; // 1 2 3 4 5 6 7 8 9
for (auto p: m) cout << p.first << "," << p.second << " ";
// bart,12345 huber,840284 jenne,34929

for (auto p: um) cout << p.first << "," << p.second << " ";
// bart,12345 jenne,34929 huber,840284

vector<int> vec2= vec;
cout << vec.size() << endl; // 9
cout << vec2.size() << endl; // 9

vector<int> vec3= move(vec);
cout << vec.size() << endl; // 0
cout << vec3.size() << endl; // 9

vec3.clear();
cout << vec3.size() << endl; // 0
\end{cpp}















