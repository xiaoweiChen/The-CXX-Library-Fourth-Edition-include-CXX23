You can create an empty string view. You can also create a string view from an existing string, character array, or string view.

The table below gives you an overview of the various ways of creating a string view.

\begin{center}
Member Functions to create and set a string view
\end{center}

% Please add the following required packages to your document preamble:
% \usepackage{longtable}
% Note: It may be necessary to compile the document several times to get a multi-page table to line up properly
\begin{longtable}[c]{|l|l|}
\hline
\textbf{Member Functions} & \textbf{Example}                              \\ \hline
\endfirsthead
%
\endhead
%
Empty string view         & std::string\_view str\_view                   \\ \hline
From a C-String           & std::string\_view str\_view2("C-string")      \\ \hline
From a string view        & std::string\_view str\_view3(str\_view2)      \\ \hline
From a C array            & std::string\_view str\_view4(arr, sizeof arr) \\ \hline
From a string\_view       & str\_view4 = str\_view3.substring(2, 3)       \\ \hline
From a string view        & std::string\_view str\_view5 = str\_view4     \\ \hline
\end{longtable}























