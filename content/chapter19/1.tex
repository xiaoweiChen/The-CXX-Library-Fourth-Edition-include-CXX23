多线程的基础是定义良好的内存模型,内存模型必须处理以下几点:

\begin{itemize}
\item 
原子操作:可以不中断地执行的操作。

\item 
部分排序操作:不能重新排序的操作序列。

\item 
操作的可见效果:保证对共享变量的操作在其他线程中可见。
\end{itemize}

C++内存模型与其前身Java内存模型有很多共同之处。

另外,C++允许打破顺序一致性,这是原子操作的默认行为。

顺序一致性提供了两个保证。

\begin{enumerate}
\item 
程序的指令按源代码顺序执行。

\item 
所有线程上的所有操作都有一个全局顺序。
\end{enumerate}















































