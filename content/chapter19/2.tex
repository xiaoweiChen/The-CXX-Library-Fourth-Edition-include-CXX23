C++有一组原子数据类型。首先是std::atomic\_flag,其次是类模板std::atomic,还可以使用std::atomic定义原子数据类型。

\mySamllsection{std::atomic\_flag}

std::atomic\_flag是一个原子布尔值,有一个清晰和固定的状态。这里将clear状态称为false,将set状态称为true。clear成员函数允许将其值设置为false。使用test\_and\_set成员函数,可以将值设置回true,并返回前一个值。没有成员函数可以查询当前值,不过这在C++20中有所改变。C++20中,std::atomic\_flag有一个测试成员函数,可以通过成员函数notify\_one、notify\_all和wait进行线程同步。

\begin{center}
std::atomic\_flag atomicFlag的所有操作
\end{center}

% Please add the following required packages to your document preamble:
% \usepackage{longtable}
% Note: It may be necessary to compile the document several times to get a multi-page table to line up properly
\begin{longtable}[c]{|l|l|}
\hline
\textbf{成员函数} & \textbf{描述}                                           \\ \hline
\endfirsthead
%
\endhead
%
atomicFlag.clear()        & 清除原子标志。                                         \\ \hline
\begin{tabular}[c]{@{}l@{}}atomicFlag.test\_and\_set()\\ atomicFlag.test()(C++20)\end{tabular} &
\begin{tabular}[c]{@{}l@{}}设置原子标志并返回旧值。\\ 返回标志的值。\end{tabular} \\ \hline
\begin{tabular}[c]{@{}l@{}}atomicFlag.notify\_one()(C++20)\\ atomicFlag.notify\_all()(C++20)\end{tabular} &
\begin{tabular}[c]{@{}l@{}}通知一个等待原子标志的线程。\\ 通知所有等待原子标志的线程。\end{tabular} \\ \hline
atomicFlag.wait(b)(C++20) & 阻塞线程,直到收到通知并且原子值发生变化。 \\ \hline
\end{longtable}

atomicFlag.test()返回atomicFlag的值而不改变它,也可以使用std::atomic\_flag进行线程同步:atomicFlag.wait()、atomicFlag.notify\_one()和atomicFlag.notify\_all()。成员函数notify\_one或notify\_all通知一个或所有等待的原子标志。atomicFlag.wait(boo)需要一个布尔值b,atomicFlag.wait(b)会阻塞直到收到通知。检查atomicFlag的值是否等于b,若不等于b则解除阻塞。

必须使用ATOMIC\_FLAG\_INIT:std::atomic\_flag标志(ATOMIC\_FLAG\_INIT)显式初始化std::atomic\_flag。C++20中,默认构造的std::atomic\_flag处于false状态。

std::atomic\_flag有一个突出的属性,它是唯一有保证的\href{https://en.wikipedia.org/wiki/Non-blocking_algorithm}{无锁}的原子变量。其他功能强大的原子变量,可以通过std::mutex锁定机制来完成相同的功能。

\mySamllsection{std::atomic}

类模板std::atomic有多种特化可用,需要头文件<atomic>。std::atomic<bool>和std::atomic<user-defined type>使用主模板。指针std::atomic<T*>提供偏特化,C++20中智能指针std::atomic<smart T*>提供整型指针std::atomic<整型指针>的全特化,C++20中浮点类型std::atomic<floating-point针>的全特化,也可以使用std::atomic定义原子数据类型。

\mySamllsection{原子变量的基本接口}

三个偏特化std::atomic<bool>、std::atomic<user-defined type>和std::atomic<smart T*>都支持基本原子接口。

% Please add the following required packages to your document preamble:
% \usepackage{longtable}
% Note: It may be necessary to compile the document several times to get a multi-page table to line up properly
\begin{longtable}[c]{|l|l|}
\hline
\textbf{成员函数} &
\textbf{描述} \\ \hline
\endfirsthead
%
\endhead
%
is\_lock\_free &
检查原子对象是否无锁。 \\ \hline
atomic\_ref\textless{}T\textgreater{}::is\_always\_lock\_free &
编译时检查原子类型是否总无锁。 \\ \hline
load &
原子地返回值。\\ \hline
operator T &
\begin{tabular}[c]{@{}l@{}}原子地返回值。\\ 相当于atom.load()。\end{tabular} \\ \hline
store &
用非原子值自动替换原子值。 \\ \hline
exchange &
用新值自动替换该值,返回旧值。 \\ \hline
\begin{tabular}[c]{@{}l@{}}compare\_exchange\_strong\\ compare\_exchange\_weak\end{tabular} &
自动比较,并交换值。 \\ \hline
\begin{tabular}[c]{@{}l@{}}notify\_one(C++20)\\ notify\_all(C++20)\end{tabular} &
\begin{tabular}[c]{@{}l@{}}通知一个原子等待操作。\\ 通知所有原子等待操作。\end{tabular} \\ \hline
wait(C++20) &
\begin{tabular}[c]{@{}l@{}}阻塞,直到收到通知为止。\\ 若与旧值相比不相等则返回\end{tabular} \\ \hline
\end{longtable}

compare\_exchange\_strong具有如下语法: bool compare\_exchange\_strong(T\& expected, T\& desired)。下面是对其行为的描述:

\begin{itemize}
\item 
若atomicValue与期望的原子比较返回true,则atomicValue在相同的原子操作中设置所需的值。

\item 
若比较返回false,则expected设置为atomicValue。
\end{itemize}

\mySamllsection{自定义原子变量std::atomic<user-defined type>}

有了类模板std::atomic,就可以定义用户定义的原子类型。

若将用户定义类型用于原子类型std::atomic<user-defined type>,则对该类型有许多实质性的限制。原子类型std::atomic<user-defined type>支持与std::atomic<bool>相同的接口。

以下是用户定义类型成为原子类型的限制:

\begin{itemize}
\item 
用户定义类型的复制赋值操作符对于其所有基类和非静态成员必须是简单数据类型。不能定义复制赋值操作符,但可以使用\href{http://en.cppreference.com/w/cpp/keyword/default}{default}从编译器请求它。

\item 
用户定义类型不能有虚成员函数或虚基类

\item 
用户定义的类型必须具有位可比性,这样才能应用C函数\href{http://en.cppreference.com/w/cpp/string/byte/memcpy}{memcpy}或\href{http://en.cppreference.com/w/cpp/string/byte/memcmp}{memcmp}
\end{itemize}

若用户定义类型的大小不大于int,那么在主流平台上都可以对std::atomic<user-defined type>使用原子操作。

\mySamllsection{原子智能指针std::atomic<smart T*> (C++20)}

std::shared\_ptr由一个控制块和资源组成。控制块是线程安全的,但对资源的访问不是。所以修改引用计数器是一个原子操作,可以保证资源会精确地删除一次,这些都是std::shared\_ptr提供的保证。使用偏特化模板std::atomic<std::shared\_ptr<T>{}>和std::atomic<std::weak\_ptr<T>{}>可以额外保证底层对象的访问是线程安全的。在2022年,原子智能指针上的所有操作都有锁。

\mySamllsection{std::atomic<floating-point type> (C++20)}

除了基本原子接口之外,std::atomic<floating-point type>还支持加法和减法。

\begin{center}
基本原子接口的其他操作
\end{center}

% Please add the following required packages to your document preamble:
% \usepackage{longtable}
% Note: It may be necessary to compile the document several times to get a multi-page table to line up properly
\begin{longtable}[c]{|l|l|}
\hline
\textbf{成员函数} & \textbf{描述}                  \\ \hline
\endfirsthead
%
\endhead
%
fetch\_add, +=            & 自动添加(减去)值。 \\ 
fetch\_sub, -=            & 返回旧值。                \\ \hline
\end{longtable}

可以使用float、double和long double类型的全特化。

\mySamllsection{std::atomic<T*>}

std::atomic<T*>是类模板std::atomic的偏特化。它的行为就像一个普通的指针T*。除了std::atomic<floating-point type>之外,std::atomic<T*>还支持前后自增或前后自减操作。

\begin{center}
std::atomic<floating-point type>的其他操作
\end{center}

% Please add the following required packages to your document preamble:
% \usepackage{longtable}
% Note: It may be necessary to compile the document several times to get a multi-page table to line up properly
\begin{longtable}[c]{|l|l|}
\hline
\textbf{成员函数} & \textbf{描述}                                          \\ \hline
\endfirsthead
%
\endhead
%
++, --                    & 递增或递减(递增前和递增后)原子变量。 \\ \hline
\end{longtable}

来看个简短的例子。

\begin{cpp}
int intArray[5];
std::atomic<int*> p(intArray);
p++;
assert(p.load() == &intArray[1]);
p+=1;
assert(p.load() == &intArray[2]);
--p;
assert(p.load() == &intArray[1]);
\end{cpp}

\mySamllsection{std::atomic<integral type>}

对于每个整型都有std::atomic的全特化std::atomic<integral type>。std::atomic< integer type>支持std::atomic<T*>或std::atomic<float -point type>支持的所有操作。此外,std::atomic<integral type>支持与、或和异或的位逻辑操作符。

\begin{center}
原子变量上的所有操作
\end{center}

% Please add the following required packages to your document preamble:
% \usepackage{longtable}
% Note: It may be necessary to compile the document several times to get a multi-page table to line up properly
\begin{longtable}[c]{|l|l|}
\hline
\textbf{成员函数} &
\textbf{描述} \\ \hline
\endfirsthead
%
\endhead
%
\begin{tabular}[c]{@{}l@{}}fetch\_or, |=\\ fetch\_and, \&=\\ fethc\_xor, \textasciicircum{}=\end{tabular} &
\begin{tabular}[c]{@{}l@{}}自动对值执行位(与、或、异或)操作。\\ 返回旧值。\end{tabular} \\ \hline
\end{longtable}

复合位赋值操作和取操作之间有细微的区别。复合按位赋值操作返回新值,fetch返回旧值。

更深入的观察可以提供更多的见解:没有原子乘法、原子除法或原子移位操作可用。这不是一个重要的限制,很少需要这些操作,并且可以很容易地实现。下面是一个原子fetch\_mult函数的示例。

\filename{使用compare\_exchange\_strong的原子乘法}

\begin{cpp}
// fetch_mult.cpp

#include <atomic>
#include <iostream>

template <typename T>
T fetch_mult(std::atomic<T>& shared, T mult){
	T oldValue = shared.load();
	while (!shared.compare_exchange_strong(oldValue, oldValue * mult));
	return oldValue;
}

int main(){
	std::atomic<int> myInt{5};
	std::cout << myInt << '\n';
	fetch_mult(myInt,5);
	std::cout << myInt << '\n';
}
\end{cpp}

第9行中的乘法只有在关系oldValue == shared成立时才会发生。我将乘法放入while循环中,以确保乘法总是发生,因为第8行中有两条指令用于读取oldValue,第9行中有两条指令用于使用它。

这是原子乘法的结果。

\myGraphic{0.8}{content/chapter19/images/2.jpg}{}

\mySamllsection{std::atomic\_ref}

此外,类模板std::atomic\_ref将原子操作应用于引用对象,原子对象的并发读写是线程安全的。引用对象的生存期必须超过atomic\_ref的生存期。std::atomic\_ref支持相同的类型和操作,就像std::atomic支持其底层类型一样。

\begin{cpp}
struct Counters {
	int a;
	int b;
};

Counter counter;
std::atomic_ref<Counters> cnt(counter);
\end{cpp}

\mySamllsection{所有原子操作}

下表显示了所有原子操作。要获得有关这些操作的进一步信息,请参阅前面关于原子数据类型的章节。

\begin{center}
所有原子操作取决于原子类型
\end{center}

% Please add the following required packages to your document preamble:
% \usepackage{longtable}
% Note: It may be necessary to compile the document several times to get a multi-page table to line up properly
\begin{longtable}[c]{|l|l|l|l|l|l|}
\hline
\textbf{成员函数} &
\textbf{atomic\_flag} &
\multicolumn{1}{c|}{\textbf{\begin{tabular}[c]{@{}c@{}}atomic\textless{}bool\textgreater\\ atomic\textless{}user\textgreater\\ atomic\textless{}smart T*\textgreater{}\end{tabular}}} &
\textbf{atomic\textless{}float\textgreater{}} &
\textbf{atomic\textless{}T*\textgreater{}} &
\textbf{atomic\textless{}int\textgreater{}} \\ \hline
\endfirsthead
%
\endhead
%
test\_and\_set                                           & yes &     &     &     &     \\ \hline
clear                                                    & yes &     &     &     &     \\ \hline
is\_lock\_free                                           &     & yes & yes & yes & yes \\ \hline
\begin{tabular}[c]{@{}c@{}}atomic\textless{}T\textgreater{}::\\is\_always\_lock\_free\end{tabular} &     & yes & yes & yes & yes \\ \hline
load                                                     &     & yes & yes & yes & yes \\ \hline
operator T                                               &     & yes & yes & yes & yes \\ \hline
store                                                    &     & yes & yes & yes & yes \\ \hline
exchange                                                 &     & yes & yes & yes & yes \\ \hline
compare\_exchange\_strong                                &     & yes & yes & yes & yes \\ \hline
compare\_exchange\_weak                                  &     & yes & yes & yes & yes \\ \hline
fetch\_add, +=                                           &     &     & yes & yes & yes \\ \hline
fetch\_sub, -=                                           &     &     & yes & yes & yes \\ \hline
fetch\_or, |=                                            &     &     &     &     & yes \\ \hline
fetch\_and, \&=                                          &     &     &     &     & yes \\ \hline
fecth\_xor, \textasciicircum{}=                          &     &     &     &     & yes \\ \hline
++, --                                                   &     &     &     & yes & yes \\ \hline
notify\_one(C++20)                                       & yes & yes & yes & yes & yes \\ \hline
notify\_all(C++20)                                       & yes & yes & yes & yes & yes \\ \hline
wait(C++20)                                              & yes & yes & yes & yes & yes \\ \hline
\end{longtable}

std::atomic<float>表示原子\href{https://en.cppreference.com/w/cpp/types/is_floating_point}{浮点数类型}, Std::atomic<int>表示原子\href{https://en.cppreference.com/w/cpp/types/is_integral}{整数类型}。













