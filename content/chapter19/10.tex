除了线程之外,C++还有异步执行工作的任务,需要头文件<future>。任务是由两个相关组件组成的工作包参数化的:promise和future,两者都通过数据通道连接。promise执行工作包并将结果放入数据通道;相关的future获取结果。两个通信端点都可以在单独的线程中运行。特别的是,future可以在以后获取结果,所以promise对结果的计算独立于相关future对结果的查询。

\begin{myTip}{将任务视为数据通道}
任务的行为类似于数据通道。promise将其结果放在数据通道中,future等着获取结果。

\myGraphic{0.6}{content/chapter19/images/13.jpg}{}
\end{myTip}

\mySamllsection{线程与任务}

线程与任务不同。

必须使用共享变量来实现创建线程和被创建线程之间的通信。任务通过其数据通道进行通信,该数据通道受到隐式保护。因此,任务不能使用像互斥锁这样的保护机制。

创建线程正在用join调用等待它的子线程。若没有结果则阻塞,可使用fut.get()获取结果。

若在创建的线程中发生异常,则创建的线程、创建者和整个进程终止。相反,promise可以向future发送异常,future必须处理该异常。

一个promise可以服务于一个或多个future。它可以发送值、异常或仅发送通知。

可以使用任务作为条件变量的安全替代。

\begin{cpp}
#include <future>
#include <thread>
...

int res;
std::thread t([&]{ res= 2000+11;});
t.join();
std::cout << res << '\n'; // 2011

auto fut= std::async([]{ return 2000+11; });
std::cout << fut.get() << '\n'; // 2011
\end{cpp}

子线程t和异步函数调用std::async计算2000和11的和。创建线程通过共享变量res从子线程t获取结果。调用std::async在发送方(promise)和接收方(future)之间创建数据通道。future使用fut.get()向数据通道请求计算结果,fut.get()会阻塞线程。

\mySamllsection{std::async}

Std::async的行为类似于异步函数调用。这个函数调用接受一个可调用对象及其参数。std::async是一个可变的模板,因此可以接受任意数量的参数。调用std::async返回一个future对象fut,通过fut.get()获取结果的句柄。还可以为std::async指定一个启动策略,可以使用启动策略显式地确定异步操作是应该在同一个线程(std::launch::deferred)中执行,还是应该在另一个线程(std::launch::async)中执行。

调用auto fut= std::async(std::launch::deferred,…)不会立即执行。调用fut.get()会惰性地启动promise。

\filename{使用std::async实现惰性和立即求值}

\begin{cpp}
// asyncLazyEager.cpp
...
#include <future>
...
using std::chrono::duration;
using std::chrono::system_clock;
using std::launch;

auto begin= system_clock::now();

auto asyncLazy= std::async(launch::deferred, []{ return system_clock::now(); });
auto asyncEager= std::async(launch::async, []{ return system_clock::now(); });
std::this_thread::sleep_for(std::chrono::seconds(1));

auto lazyStart= asyncLazy.get() - begin;
auto eagerStart= asyncEager.get() - begin;

auto lazyDuration= duration<double>(lazyStart).count();
auto eagerDuration= duration<double>(eagerStart).count();

std::cout << lazyDuration << " sec"; // 1.00018 sec.
std::cout << eagerDuration << " sec"; // 0.00015489 sec.
\end{cpp}

程序的输出显示,与future的asyncLazy相关联的promise比与future的asyncEager相关联的promise晚一秒执行。一秒是创建者在future asynlazy请求其结果之前的休眠时间。

\begin{myTip}{std::async应该是首选}
C++运行时确定std::async是否在单独的线程中执行。C++运行时的决定可能取决于内核的数量、系统的利用率或工作包的大小。
\end{myTip}

\mySamllsection{std::packaged\_task}

std::packaged\_task能够为可调用对象构建一个简单的包装器,稍后可以在单独的线程上执行。

因此,四个步骤是必要的。

\begin{enumerate}[label=\Roman*.]
\item 
打好工作包:

\begin{cpp}
std::packaged_task<int(int, int)> sumTask([](int a, int b){ return a+b; });
\end{cpp}

\item
创建future:

\begin{cpp}
std::future<int> sumResult= sumTask.get_future();
\end{cpp}
 
\item 
执行计算:

\begin{cpp}
sumTask(2000, 11);
\end{cpp}

\item 
查询结果:

\begin{cpp}
sumResult.get();
\end{cpp}

\end{enumerate}

可以在单独的线程中移动std::package\_task或std::future。

\filename{std::packaged\_task}

\begin{cpp}
// packaged_task.cpp
...
#include <future>
...

using namespace std;

struct SumUp{
	int operator()(int beg, int end){
		for (int i= beg; i < end; ++i ) sum += i;
		return sum;
	}
	private:
	int beg;
	int end;
	int sum{0};
};

SumUp sumUp1, sumUp2;

packaged_task<int(int, int)> sumTask1(sumUp1);
packaged_task<int(int, int)> sumTask2(sumUp2);

future<int> sum1= sumTask1.get_future();
future<int> sum2= sumTask2.get_future();

deque< packaged_task<int(int, int)>> allTasks;
allTasks.push_back(move(sumTask1));
allTasks.push_back(move(sumTask2));

int begin{1};
int increment{5000};
int end= begin + increment;

while (not allTasks.empty()){
	packaged_task<int(int, int)> myTask= move(allTasks.front());
	allTasks.pop_front();
	thread sumThread(move(myTask), begin, end);
	begin= end;
	end += increment;
	sumThread.detach();
}

auto sum= sum1.get() + sum2.get();
cout << sum; // 50005000
\end{cpp}

promise(std::packaged\_task)移动到std::deque allTasks中。程序在while循环中遍历所有承诺。每个promise在其线程中运行,并在后台执行其添加(sumThread.detach())。结果是1到100000之间所有数字的和。

\mySamllsection{std::promise 和 std::future}

std::promise和std::future对提供了对任务的完全控制。

\begin{center}
promise prom的成员函数
\end{center}

% Please add the following required packages to your document preamble:
% \usepackage{longtable}
% Note: It may be necessary to compile the document several times to get a multi-page table to line up properly
\begin{longtable}[c]{|l|l|}
\hline
\textbf{成员函数} & \textbf{描述} \\ \hline
\endfirsthead
%
\endhead
%
\begin{tabular}[c]{@{}l@{}}prom.swap(prom2) \\ std::swap(prom, prom2)\end{tabular} & 交换promises。                                           \\ \hline
prom.get\_future()       & 返回future。  \\ \hline
prom.set\_value(val)     & 设置值。      \\ \hline
prom.set\_exception(ex)  & 设置异常。  \\ \hline
prom.set\_value\_at\_thread\_exit(val)                                                & 存储该值,并使其在promise退出时准备好。     \\ \hline
prom.set\_exception\_at\_thread\_exit(ex)                                             & 存储异常,并使其在promise退出时准备好。 \\ \hline
\end{longtable}

若promise不止一次设置值或异常,则抛出std::future\_error异常。

\begin{center}
future fut的成员函数
\end{center}

% Please add the following required packages to your document preamble:
% \usepackage{longtable}
% Note: It may be necessary to compile the document several times to get a multi-page table to line up properly
\begin{longtable}[c]{|l|l|}
\hline
\textbf{成员函数} & \textbf{描述}           \\ \hline
\endfirsthead
%
\endhead
%
fut.share()              & 返回std::shared\_future。 \\ \hline
fut.get()   & 返回结果,该结果可以是一个值或异常。                  \\ \hline
fut.valid() & 检查结果是否可用。调用fut.get()后返回false。 \\ \hline
fut.wait()               & 等待结果。          \\ \hline
fut.wait\_for(relTime)   & 等待结果的一段时间。 \\ \hline
fut.wait\_until(absTime) & 等待结果的直到某个绝对时间点。        \\ \hline
\end{longtable}

若future fut不止一次请求结果,则抛出std::future\_error异常。future可通过fut.share()创建共享future。共享future与他们的promise相关联,并且可以独立地要求结果。共享future与future具有相同的接口。

下面是promise和future的用法。

\filename{Promise 和 future}

\begin{cpp}
// promiseFuture.cpp
...
#include <future>
...

void product(std::promise<int>&& intPromise, int a, int b){
	intPromise.set_value(a*b);
}

int a= 20;
int b= 10;

std::promise<int> prodPromise;
std::future<int> prodResult= prodPromise.get_future();

std::thread prodThread(product, std::move(prodPromise), a, b);
std::cout << "20*10= " << prodResult.get(); // 20*10= 200
\end{cpp}

promise prodPromise移动到一个单独的线程中,并执行它的计算。future通过prodResult.get()获取结果。

\mySamllsection{同步}

future fut可以通过调用fut.wait()与它关联的promise同步。与条件变量相反,不需要锁和互斥锁,并且不可能出现伪唤醒和未唤醒。

\filename{同步任务}

\begin{cpp}
// promiseFutureSynchronise.cpp
...
#include <future>
...

void doTheWork(){
	std::cout << "Processing shared data." << '\n';
}

void waitingForWork(std::future<void>&& fut){
	std::cout << "Worker: Waiting for work." <<
	'\n';
	fut.wait();
	doTheWork();
	std::cout << "Work done." << '\n';
}

void setDataReady(std::promise<void>&& prom){
	std::cout << "Sender: Data is ready." <<
	'\n';
	prom.set_value();
}

std::promise<void> sendReady;
auto fut= sendReady.get_future();

std::thread t1(waitingForWork, std::move(fut));
std::thread t2(setDataReady, std::move(sendReady));
\end{cpp}

promise prom.set\_value()的调用唤醒future,然后可以执行其工作。

\myGraphic{0.6}{content/chapter19/images/14.jpg}{}













