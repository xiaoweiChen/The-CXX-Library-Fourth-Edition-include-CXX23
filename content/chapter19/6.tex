
使用关键字thread\_local,就拥有了线程本地数据,也称为线程本地存储。每个线程都有自己的数据副本。线程本地数据的行为类似于静态变量。它们是在第一次使用时创建的,其生命周期与线程的生命周期绑定在一起。

\filename{线程本地数据}

\begin{cpp}
	// threadLocal.cpp
	...
	std::mutex coutMutex;
	thread_local std::string s("hello from ");
	
	void addThreadLocal(std::string const& s2){
		s+= s2;
		std::lock_guard<std::mutex> guard(coutMutex);
		std::cout << s << '\n';
		std::cout << "&s: " << &s << '\n';
		std::cout << '\n';
	}
	
	std::thread t1(addThreadLocal, "t1");
	std::thread t2(addThreadLocal, "t2");
	std::thread t3(addThreadLocal, "t3");
	std::thread t4(addThreadLocal, "t4");
\end{cpp}

每个线程都有一个thread\_local字符串的副本,每个字符串都独立地修改它的字符串,并且每个字符串都有它唯一的地址:

\myGraphic{0.4}{content/chapter19/images/9.jpg}{}












































