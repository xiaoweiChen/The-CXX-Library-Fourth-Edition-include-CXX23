全局函数std::begin、std::end、std::prev、std::next、std::distance和std::advance使处理迭代器变得容易得多。只有std::prev函数需要双向迭代器。所有函数都需要头文件<iterator>,下面表格是一个简单的概述。

\begin{center}
迭代器的辅助函数
\end{center}

% Please add the following required packages to your document preamble:
% \usepackage{longtable}
% Note: It may be necessary to compile the document several times to get a multi-page table to line up properly
\begin{longtable}[c]{|l|l|}
\hline
\textbf{辅助函数} & \textbf{描述}                                      \\ \hline
\endfirsthead
%
\endhead
%
std::begin(cont)         & 向容器cont返回一个起始迭代器。           \\ \hline
std::end(cont)           & 向容器cont返回一个结束迭代器。            \\ \hline
std::rbegin(cont)        & 向容器cont返回一个反向起始迭代器。   \\ \hline
std::rend(cont)          & 向容器cont返回一个反向结束迭代器。     \\ \hline
std::cbegin(cont)        & 向容器cont返回一个常量起始迭代器。  \\ \hline
std::cend(cont)          & 向容器cont返回一个常量结束迭代器。   \\ \hline
std::crbegin(cont)      & 向容器cont返回一个反向常量起始迭代器。 \\ \hline
std::crend(cont)        & 向容器cont返回一个反向常量结束迭代器。   \\ \hline
std::prev(it)            & 返回一个迭代器,该迭代器指向它之前的位置。 \\ \hline
std::next(it)            & 返回一个迭代器,该迭代器指向它之后的位置。 \\ \hline
std::distance(fir, sec) & 返回fir和sec之间的元素个数。              \\ \hline
std::advance(it, n)      & 将迭代器前进或后退n个位置。                 \\ \hline
\end{longtable}

下面是辅助函数的应用。


\filename{迭代器的辅助函数}

\begin{cpp}
// iteratorUtilities.cpp
...
#include <iterator>
...
using std::cout;

std::unordered_map<std::string, int> myMap{{"Rainer", 1966}, {"Beatrix", 1966},
											{"Juliette", 1997}, {"Marius", 1999}};
	
for (auto m: myMap) cout << "{" << m.first << "," << m.second << "} ";
	// {Juliette,1997},{Marius,1999},{Beatrix,1966},{Rainer,1966}
	
auto mapItBegin= std::begin(myMap);
cout << mapItBegin->first << " " << mapItBegin->second; // Juliette 1997

auto mapIt= std::next(mapItBegin);
cout << mapIt->first << " " << mapIt->second; // Marius 1999
cout << std::distance(mapItBegin, mapIt); // 1

std::array<int, 10> myArr{0, 1, 2, 3, 4, 5, 6, 7, 8, 9};
for (auto a: myArr) std::cout << a << " "; // 0 1 2 3 4 5 6 7 8 9

auto arrItEnd= std::end(myArr);
auto arrIt= std::prev(arrItEnd);

cout << *arrIt << '\n'; // 9

std::advance(arrIt, -5);
cout << *arrIt; // 4
\end{cpp}


















