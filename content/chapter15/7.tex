
std::regex\_replace and std::match\_results.format in combination with capture groups enables you to format text. You can use a format string together with a placeholder to insert the value.

Here are both possibilities:

\filename{Formatting with regex}

\begin{cpp}
// format.cpp
...
#include <regex>
...

std::string future{"Future"};
const std::string unofficial{"unofficial, C++0x"};
const std::string official{"official, C++11"};

std::regex regValues{"(.*),(.*)"};
std::string standardText{"The $1 name of the new C++ standard is $2."};
std::string textNow= std::regex_replace(unofficial, regValues, standardText);
std::cout << textNow << '\n';
			// The unofficial name of the new C++ standard is C++0x.

std::smatch smatch;
if (std::regex_match(official, smatch, regValues)){
	std::cout << smatch.str(); // official,C++11
	std::string textFuture= smatch.format(standardText);
	std::cout << textFuture << '\n';
} // The official name of the new C++ standard is C++11.
\end{cpp}

In the function call std::regex\_replace(unoffical, regValues, standardText), the text matching the first and second capture group of the regular expression regValues is extracted from the string unofficial. The placeholders \$1 and \$2 in the text standardText are then replaced by the extracted values. The strategy of smatch.format(standardTest) is similar, but there is a difference: 

The creation of the search results smatch is separated from their usage when formatting the string.

In addition to capture groups, C++ supports additional format escape sequences. You can use them in format strings:

\begin{center}
Format escape sequences
\end{center}

% Please add the following required packages to your document preamble:
% \usepackage{longtable}
% Note: It may be necessary to compile the document several times to get a multi-page table to line up properly
\begin{longtable}[c]{|l|l|}
\hline
\textbf{Format escape sequence} & \textbf{Description}                        \\ \hline
\endfirsthead
%
\endhead
%
\$\&                            & Returns the total match(0th capture group). \\ \hline
$$                              & Returns \$.                                 \\ \hline
\$`(backward tic)               & Returns the text before the total match.    \\ \hline
\$\^{A}'(forward tic) & Returns the text after the total match. \\ \hline
'\$ i'                          & Returns the ith capture group.              \\ \hline
\end{longtable}























