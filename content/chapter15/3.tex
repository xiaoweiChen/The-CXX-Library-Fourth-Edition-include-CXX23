

std::match\_results类型的对象是std::regex\_match或std::regex\_search的结果。Std::match\_results是一个序列容器,至少有一个std::sub\_match对象的捕获组,std::sub\_match对象是字符序列。

\begin{myNotic}{什么是捕获组?}
捕获组允许它进一步分析正则表达式中的搜索结果,由一对括号()定义。正则表达式((a+)(b+)(c+))有四个捕获组:((a+)(b+)(c+))、(a+)、(b+)和(c+),总结果是第0个捕获组。
\end{myNotic}

C++有四种类型的std::match\_results类型的同义词:

\begin{cpp}
typedef match_results<const char*> cmatch;
typedef match_results<const wchar_t*> wcmatch;
typedef match_results<string::const_iterator> smatch;
typedef match_results<wstring::const_iterator> wsmatch;
\end{cpp}

搜索结果std::smatch smatch具有强大的接口。

\begin{center}
std::smatch的接口
\end{center}

% Please add the following required packages to your document preamble:
% \usepackage{longtable}
% Note: It may be necessary to compile the document several times to get a multi-page table to line up properly
\begin{longtable}[c]{|l|l|}
\hline
\textbf{成员函数}            & \textbf{描述}                                       \\ \hline
\endfirsthead
%
\endhead
%
smatch.size()                       & 返回捕获组的数目。                      \\ \hline
smatch.empty()                      & 若搜索结果有捕获组,则返回。          \\ \hline
smatch{[}i{]}                       & 返回第i个捕获组。                            \\ \hline
smatch.length(i)                    & 返回第i个捕获组的长度。              \\ \hline
smatch.position(i)                  & 返回第i个捕获组的位置。            \\ \hline
smatch.str(i)                       & 以字符串形式返回第i个捕获组。                  \\ \hline
smatch.prefix() and smatch.suffix() & 返回捕获组前后的字符串。     \\ \hline
smatch.begin() and smatch.end()     & 返回捕获组的开始和结束迭代器。 \\ \hline
smatch.format(...)                  & 为输出格式化std::smatch对象。                \\ \hline
\end{longtable}

下面的代码展示了针对不同正则表达式,前四个捕获组的输出。

\filename{捕获组}

\begin{cpp}
// captureGroups.cpp
...
#include<regex>
...
using namespace std;

void showCaptureGroups(const string& regEx, const string& text){
	regex rgx(regEx);
	smatch smatch;
	if (regex_search(text, smatch, rgx)){
		cout << regEx << text << smatch[0] << " " << smatch[1]
		<< " "<< smatch[2] << " " << smatch[3] << endl;
	}
}

showCaptureGroups("abc+", "abccccc");
showCaptureGroups("(a+)(b+)", "aaabccc");
showCaptureGroups("((a+)(b+))", "aaabccc");
showCaptureGroups("(ab)(abc)+", "ababcabc");
...
\end{cpp}

% Please add the following required packages to your document preamble:
% \usepackage{longtable}
% Note: It may be necessary to compile the document several times to get a multi-page table to line up properly
\begin{longtable}[c]{|l|l|l|l|l|l|}
\hline
正则表达式       & 文本     & smatch{[}0{]} & smatch{[}1{]} & smatch{[}2{]} & smatch{[}3{]} \\ \hline
\endfirsthead
%
\endhead
%
abc+           & abccccc  & abccccc       &               &               &               \\ \hline
(a+)(b+)(c+)   & aaabccc  & aaabccc       & aaa           & b             & ccc           \\ \hline
((a+)(b+)(c+)) & aaabccc  & aaabccc       & aaabccc       & aaa           & b             \\ \hline
(ab)(abc)+     & ababcabc & ababcabc      & ab            & abc           &              \\ \hline
\end{longtable}


\mySamllsection{std::sub\_match}

捕获组的类型为std::sub\_match。与std::match\_results一样,C++定义了以下四个同义类型。

\begin{cpp}
typedef sub_match<const char*> csub_match;
typedef sub_match<const wchar_t*> wcsub_match;
typedef sub_match<string::const_iterator> ssub_match;
typedef sub_match<wstring::const_iterator> wssub_match;
\end{cpp}

可以进一步分析捕获组的内容

\begin{center}
std::sub\_match
\end{center}

% Please add the following required packages to your document preamble:
% \usepackage{longtable}
% Note: It may be necessary to compile the document several times to get a multi-page table to line up properly
\begin{longtable}[c]{|l|l|}
\hline
\textbf{成员函数} & \textbf{描述}                     \\ \hline
\endfirsthead
%
\endhead
%
cap.matched()            & 指示匹配是否成功。  \\ \hline
cap.first() and cap.end() & 返回字符序列的开始和结束迭代器。    \\ \hline
cap.length()             & 返回捕获组的长度。 \\ \hline
cap.str()                & 以字符串形式返回捕获组。     \\ \hline
cap.compare(other)        & 将当前捕获组与其他捕获组进行比较。 \\ \hline
\end{longtable}

下面的代码片段显示了搜索结果std::match\_results和其捕获组std::sub\_match之间的影响。

\filename{std::sub\_match}

\begin{cpp}
// subMatch.cpp
...
#include <regex>
...
using std::cout;
std::string privateAddress="192.168.178.21";
std::string regEx(R"((\d{1,3})\.(\d{1,3})\.(\d{1,3})\.(\d{1,3}))");
std::regex rgx(regEx);
std::smatch smatch;
if (std::regex_match(privateAddress, smatch, rgx)){
	for (auto cap: smatch){
		cout << "capture group: " << cap << '\n';
		if (cap.matched){
			std::for_each(cap.first, cap.second, [](int v){
				cout << std::hex << v << " ";});
			cout << '\n';
		}
	}
}
...
\end{cpp}

\begin{shell}
capture group: 192.168.178.21
31 39 32 2e 31 36 38 2e 31 37 38 2e 32 31

capture group: 192
31 39 32

capture group: 168
31 36 38

capture group: 178
31 37 38

capture group: 21
32 31
\end{shell}

正则表达式regEx表示IPv4地址,regEx使用捕获组提取地址的组件。最后,捕获组和ASCII字符以十六进制值显示。















