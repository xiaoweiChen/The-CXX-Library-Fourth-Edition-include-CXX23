

使用std::regex\_iterator和std::regex\_token\_iterator遍历匹配的文本非常方便。std::regex\_iterator支持匹配及其捕获组。std::regex\_token\_iterator支持更多。可以处理每个捕获的组件,使用负索引使它能够访问匹配之间的文本。

\mySamllsection{std::regex\_iterator}

C++为std::regex\_iterator定义了以下四个同义类型。

\begin{cpp}
typedef cregex_iterator regex_iterator<const char*>
typedef wcregex_iterator regex_iterator<const wchar_t*>
typedef sregex_iterator regex_iterator<std::string::const_iterator>
typedef wsregex_iterator regex_iterator<std::wstring::const_iterator>
\end{cpp}

可以使用std::regex\_iterator来计算文本中单词的出现次数:

\filename{std::regex\_iterator}

\begin{cpp}
// regexIterator.cpp
...
#include <regex>
...

using std::cout;
std::string text{"That's a (to me) amazingly frequent question. It may be the most freque\
ntly asked question. Surprisingly, C++11 feels like a new language: The pieces just fit t\
ogether better than they used to, and I find a higher-level style of programming more nat\
ural than before and as efficient as ever." };

std::regex wordReg{R"(\w+)"};
std::sregex_iterator wordItBegin(text.begin(), text.end(), wordReg);
const std::sregex_iterator wordItEnd;
std::unordered_map<std::string, std::size_t> allWords;
for (; wordItBegin != wordItEnd; ++wordItBegin){
	++allWords[wordItBegin->str()];
}
for (auto wordIt: allWords) cout << "(" << wordIt.first << ":"
								        << wordIt.second << ")";
					// (as:2)(of:1)(level:1)(find:1)(ever:1)(and:2)(natural:1) ...
\end{cpp}

一个单词至少包含一个字符(\verb|\|w+),这个正则表达式用于定义开始迭代器wordItBegin和结束迭代器wordItEnd。匹配项的迭代发生在for循环中,每个单词增加计数器:++allWords[wordItBegin]->str()]。若有单词不存在于allWords中,则创建一个计数器等于1的单词。

\mySamllsection{std::regex\_token\_iterator}

C++为std::regex\_token\_iterator定义了以下四个同义类型。

\begin{cpp}
typedef cregex_iterator regex_iterator<const char*>
typedef wcregex_iterator regex_iterator<const wchar_t*>
typedef sregex_iterator regex_iterator<std::string::const_iterator>
typedef wsregex_iterator regex_iterator<std::wstring::const_iterator>
\end{cpp}

std::regex\_token\_iterator能够使用索引显式地指定感兴趣的捕获组。若不指定索引,将获得所有捕获组,也可以使用它们各自的索引请求特定的捕获组。-1索引是特殊的:可以使用-1来定位匹配之间的文本:

\filename{std::regex\_token\_iterator}

\begin{cpp}
// tokenIterator.cpp
...
using namespace std;

std::string text{"Pete Becker, The C++ Standard Library Extensions, 2006:"
				 "Nicolai Josuttis, The C++ Standard Library, 1999:"
				 "Andrei Alexandrescu, Modern C++ Design, 2001"};
	
regex regBook(R"((\w+)\s(\w+),([\w\s\+]*),(\d{4}))");
sregex_token_iterator bookItBegin(text.begin(), text.end(), regBook);

const sregex_token_iterator bookItEnd;
while (bookItBegin != bookItEnd){
	cout << *bookItBegin++ << endl;
} // Pete Becker,The C++ Standard Library Extensions,2006
  // Nicolai Josuttis,The C++ Standard Library,1999

sregex_token_iterator bookItNameIssueBegin(text.begin(), text.end(),
										   regBook, {{2,4}});
const sregex_token_iterator bookItNameIssueEnd;
\end{cpp}










