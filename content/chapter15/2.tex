

正则表达式类型的对象是类模板template <class charT,class traits= regex\_traits <charT>{}>类basic\_regex的实例,由字符类型和特征类参数化,特征类定义了正则语法属性的解释。C++中有两种同义类型:

\begin{cpp}
typedef basic_regex<char> regex;
typedef basic_regex<wchar_t> wregex;
\end{cpp}

可以进一步自定义正则表达式类型的对象,可以指定所使用的语法或调整语法。如前所述,C++支持基本、扩展、awk、grep和egrep语法。由std::regex\_constants::icase标志限定的正则表达式不区分大小写。若要采用该语法,则必须显式指定语法。

\filename{指定语法}

\begin{cpp}
// regexGrammar.cpp
...
#include <regex>
...
using std::regex_constants::ECMAScript;
using std::regex_constants::icase;

std::string theQuestion="C++ or c++, that's the question.";
std::string regExprStr(R"(c\+\+)");

std::regex rgx(regExprStr);
std::smatch smatch;

if (std::regex_search(theQuestion, smatch, rgx)){
	std::cout << "case sensitive: " << smatch[0]; // c++
}

std::regex rgxIn(regExprStr, ECMAScript|icase);
if (std::regex_search(theQuestion, smatch, rgxIn)){
	std::cout << "case insensitive: " << smatch[0]; // C++
}
\end{cpp}

若使用区分大小写的正则表达式rgx,则在文本theQuestion中搜索的结果是“c++”。若使用不区分大小写的正则表达式rgxIn,则不是这种情况。现在,得到了匹配的字符串是“C++”。



























