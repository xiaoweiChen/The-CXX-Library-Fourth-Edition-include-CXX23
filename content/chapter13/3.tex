The number of elements a string has (str.size()) is in general, smaller than the number of elements for which space is reserved: str.capacity(). Therefore if you add elements to a string, there will not automatically be new memory allocated. std:max\_size() return how many elements a string can maximal have. For the three member functions the following relation holds: str.size() <= str.capacity() <= str.max\_size().

The following table shows the member functions for dealing with the memory management of the string.

\begin{center}
Member Functions to create and delete a string
\end{center}


% Please add the following required packages to your document preamble:
% \usepackage{longtable}
% Note: It may be necessary to compile the document several times to get a multi-page table to line up properly
\begin{longtable}[c]{|l|l|}
\hline
\textbf{Member Functions} & \textbf{Description}                                  \\ \hline
\endfirsthead
%
\endhead
%
str.empty()               & Checks if str has elements.                           \\ \hline
str.size(), str.length()  & Number of elements of the str.                        \\ \hline
str.capacity()            & Number of elements str can have without reallocation. \\ \hline
str.max\_size()           & Number of elements str can maximal have.              \\ \hline
std.resize(n)             & Resize str to n elements.                             \\ \hline
str.resize\_and\_overwrite(n, op) & Resize str to n elements and applies the operation op to its elements. \\ \hline
str.reserve(n)            & Reserve memory for a least n elements.                \\ \hline
std.shrink\_to\_fit()     & Adjusts the capacity of the string to it's size.      \\ \hline
\end{longtable}

The request str.shrink\_to\_fit() is, as in the case of std::vector, non-binding.

\filename{Size versus capacity}

\begin{cpp}
// stringSizeCapacity.cpp
...
#include <string>
...

void showStringInfo(const std::string& s){
	std::cout << s << ": ";
	std::cout << s.size() << " ";
	std::cout << s.capacity() << " ";
	std::cout << s.max_size() << " ";
}

std::string str;
showStringInfo(str); // "": 0 0 4611686018427387897

str +="12345";
showStringInfo(str); // "12345": 5 5 4611686018427387897

str.resize(30);
showStringInfo(str); // "12345": 30 30 4611686018427387897

str.reserve(1000);
showStringInfo(str); // "12345": 30 1000 4611686018427387897

str.shrink_to_fit();
showStringInfo(str); // "12345": 30 30 4611686018427387897
\end{cpp}










































