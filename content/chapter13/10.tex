
字符串有很多可以修改的操作。str.assign将一个新字符串赋给字符串str,使用str.swap可以交换两个字符串。要从字符串中删除一个字符,请使用str.pop\_back或str.erase。相反,str.clear或str.erase删除整个字符串。要在字符串中添加新字符,请使用+=、std.append或str.push\_back,可以使用str.insert插入新字符或使用str.replace替换字符。

\begin{center}
用于修改字符串的函数
\end{center}

% Please add the following required packages to your document preamble:
% \usepackage{longtable}
% Note: It may be necessary to compile the document several times to get a multi-page table to line up properly
\begin{longtable}[c]{|l|l|}
\hline
\textbf{成员函数}  & \textbf{描述}                                 \\ \hline
\endfirsthead
%
\endhead
%
str=str2                   & 将str2赋值给str。                                 \\ \hline
str.assign(...)            & 将一个新字符串赋给str。                         \\ \hline
str.swap(str2)             & 交换str和str2。                                  \\ \hline
str.pop\_back()            & 从str中移除最后一个字符。                 \\ \hline
str.erase(...)             & 从str中删除字符。                         \\ \hline
str.clear()                & 清除str中的字符。                                      \\ \hline
str.append(...)            & 向str追加字符。                           \\ \hline
str.push\_back(s)          & 将字符s追加到str。                     \\ \hline
str.insert(pos, ...)       & 在str中插入从pos开始的字符。           \\ \hline
str.replace(pos, len, ...) & 替换str中从pos开始的len字符 \\ \hline
\end{longtable}

这些操作在许多重载版本中都是可用的,成员函数str.assign、str.append、str.insert和str.replace非常相似。这四个函数都可以用——字符串和子字符串、字符、C字符串、C字符串数组、范围和初始化列表。str.erase可以擦除单个字符、范围和从给定位置开始的多个字符。

下面的代码片段显示了许多变体。简单起见,只显示字符串修改的效果:

\filename{修改字符串}

\begin{cpp}
// stringModification.cpp
...
#include <string>
...

std::string str{"New String"};
std::string str2{"Other String"};

str.assign(str2, 4, std::string::npos); // r String
str.assign(5, '-'); // -----

str= {"0123456789"};
str.erase(7, 2); // 01234569
str.erase(str.begin()+2, str.end()-2); // 012
str.erase(str.begin()+2, str.end()); // 01
str.pop_back(); // 0
str.erase(); //

str= "01234";
str+= "56"; // 0123456
str+= '7'; // 01234567
str+= {'8', '9'}; // 0123456789
str.append(str); // 01234567890123456789
str.append(str, 2, 4); // 012345678901234567892345
str.append(3, '0'); // 012345678901234567892345000
str.append(str, 10, 10); // 01234567890123456789234500001234567989
str.push_back('9'); // 012345678901234567892345000012345679899

str= {"345"};
str.insert(3, "6789"); // 3456789
str.insert(0, "012"); // 0123456789

str= {"only for testing purpose."};
str.replace(0, 0, "0"); // 0nly for testing purpose.
str.replace(0, 5, "Only", 0, 4); // Only for testing purpose.
str.replace(16, 8, ""); // Only for testing.

str.replace(4, 0, 5, 'y'); // Onlyyyyyy for testing.
str.replace(str.begin(), str.end(), "Only for testing purpose.");
										// Only for testing purpose.
str.replace(str.begin()+4, str.end()-8, 10, '#');
										// Only#############purpose.
\end{cpp}











































