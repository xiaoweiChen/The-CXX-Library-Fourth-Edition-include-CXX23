C++ offers many member functions to create strings from C or C++-strings. Under the hood, a C string is always involved for creating a C++-string. That changes with C++14, because the new C++ standard support C++-string literals: std::string str{"string"s}. The C string literals "string literal" becomes with the suffix s a C++-string literal: "string literal"s.

The table gives an overview of the member functions to create and delete a C++-string.

\begin{center}
Member Functions to create and delete a string
\end{center}

% Please add the following required packages to your document preamble:
% \usepackage{longtable}
% Note: It may be necessary to compile the document several times to get a multi-page table to line up properly
\begin{longtable}[c]{|l|l|}
\hline
\textbf{Member Functions}        & \textbf{Example}                        \\ \hline
\endfirsthead
%
\endhead
%
Default                          & std::string str                         \\ \hline
Copies from a C++-string         & std::string str(oth)                    \\ \hline
Moves from a C++-string          & std::string str(std::move(oth))         \\ \hline
From the range of a C++-string   & std::string(oth.begin(), oth.end())     \\ \hline
From a substring of a C++-string & std::string(oth, otherIndex)            \\ \hline
From a substring of a C++-string & std::string(oth, otherIndex, strlen)    \\ \hline
From a C string                  & std::string str("c-string")             \\ \hline
From a C array                   & std::string str("c-array", len)         \\ \hline
From characters                  & std::string str(num, 'c')               \\ \hline
From an initializer list         & std::string str(\{'a', 'b', 'c', 'd'\}) \\ \hline
From a substring                 & str = other.substring(3, 10)            \\ \hline
Destructor                       & str.$\sim$string()                      \\ \hline
\end{longtable}

\filename{Creation of a string}

\begin{cpp}
// stringConstructor.cpp
...
#include <string>
...
std::string defaultString;
std::string other{"123456789"};
std::string str1(other); // 123456789
std::string tmp(other); // 123456789
std::string str2(std::move(tmp)); // 123456789
std::string str3(other.begin(), other.end()); // 123456789
std::string str4(other, 2); // 3456789
std::string str5(other, 2, 5); // 34567
std::string str6("123456789", 5); // 12345
std::string str7(5, '1'); // 11111
std::string str8({'1', '2', '3', '4', '5'}); // 12345
std::cout << str6.substr(); // 12345
std::cout << str6.substr(1); // 2345
std::cout << str6.substr(1, 2); // 23
\end{cpp}































