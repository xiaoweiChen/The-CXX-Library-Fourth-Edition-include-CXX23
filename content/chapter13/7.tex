A string can read from an input stream via >> and write to an output stream via <{}<.

The global function getline empowers you to read from an input stream line by line until the end-offile character.

There are four variations of the getline function available. The first two arguments are the input stream is and the string line holding the line read. Optionally you can specify a special line separator.

The function returns by reference to the input stream.

\begin{cpp}
istream& getline (istream& is, string& line, char delim);
istream& getline (istream&& is, string& line, char delim);
istream& getline (istream& is, string& line);
istream& getline (istream&& is, string& line);
\end{cpp}

getline consumes the whole line, including empty spaces. Only the line separator is ignored. The function needs the header <string>.

\filename{Input and output with strings}

\begin{cpp}
// stringInputOutput.cpp
...
#include <string>
...

std::vector<std::string> readFromFile(const char* fileName){
	std::ifstream file(fileName);
	if (!file){
		std::cerr << "Could not open the file " << fileName << ".";
		exit(EXIT_FAILURE);
	}
	std::vector<std::string> lines;
	std::string line;
	while (getline(file , line)) lines.push_back(line);
	return lines;
}

std::string fileName;
std::cout << "Your filename: ";
std::cin >> fileName;
std::vector<std::string> lines= readFromFile(fileName.c_str());
int num{0};
for (auto line: lines) std::cout << ++num << ": " << line << '\n';
\end{cpp}

The program displays the lines of an arbitrary file, including their line number. The expression std::cin >{}> fileName reads the file name. The function readFromFile reads with getline all file lines and pushes them onto the vector.


































