
C++ offers the ability to search in a string in many variations. Each variation exists in various overloaded forms.

\begin{myTip}{Search is called find}
Odd enough, the algorithms for searching in a string start with the name find. If the search was successful, you get the index of type std::string::size\_type. If not, you get the constant std::string::npos. The first character has index 0.
\end{myTip}

The find algorithms support:

\begin{itemize}
\item 
search for a character, a C String, or a C++-string,

\item 
search for a character from a C or C++-string,

\item 
search forward and backward,

\item 
search positive (does contain) or negative(does not contain) for characters from a C or C++- string,

\item 
start the search at an arbitrary position in the string.
\end{itemize}

The arguments of all six variations of the find functions follow a similar pattern. The first argument is the text you are searching for. The second argument holds the start position of the search, and the third the number of characters starting from the second argument.

Here are the six variations.

\begin{center}
Find variations of the string
\end{center}

% Please add the following required packages to your document preamble:
% \usepackage{longtable}
% Note: It may be necessary to compile the document several times to get a multi-page table to line up properly
\begin{longtable}[c]{|l|l|}
\hline
\textbf{Member Functions} & \textbf{Description}                                                      \\ \hline
\endfirsthead
%
\endhead
%
str.find(...)             & Returns the first position of a character, a C or C++-string in str.      \\ \hline
str.rfind(...)            & Returns the last position of a character, a C or C++-string in str.       \\ \hline
str.find\_first\_of(...)  & Returns the first position of a char acter from a C or C++-string in str. \\ \hline
str.find\_last\_of(...)   & Returns the last position of a character from a C or C++-string in str.   \\ \hline
str.find\_first\_not\_of(...) & Returns the first position of a character in str, which is not from a C or C++-string. \\ \hline
std.find\_last\_not\_of(...)  & Returns the last position of a character in str, which is not from a C or C++-string.  \\ \hline
\end{longtable}

\filename{Find(search) in a string}

\begin{cpp}
// stringFind.cpp
...
#include <string>
...
std::string str;
auto idx= str.find("no");
if (idx == std::string::npos) std::cout << "not found"; // not found
str= {"dkeu84kf8k48kdj39kdj74945du942"};
std::string str2{"84"};
std::cout << str.find('8'); // 4
std::cout << str.rfind('8'); // 11
std::cout << str.find('8', 10); // 11
std::cout << str.find(str2); // 4
std::cout << str.rfind(str2); // 4
std::cout << str.find(str2, 10); // 18446744073709551615
str2="0123456789";
std::cout << str.find_first_of("678"); // 4
std::cout << str.find_last_of("678"); // 20
std::cout << str.find_first_of("678", 10); // 11
std::cout << str.find_first_of(str2); // 4
std::cout << str.find_last_of(str2); // 29
std::cout << str.find_first_of(str2, 10); // 10
std::cout << str.find_first_not_of("678"); // 0
std::cout << str.find_last_not_of("678"); // 29
std::cout << str.find_first_not_of("678", 10); // 10
std::cout << str.find_first_not_of(str2); // 0
std::cout << str.find_last_not_of(str2); // 26
std::cout << str.find_first_not_of(str2, 10); // 12
\end{cpp}

The call std::find(str2, 10) returns std::string::npos. If I display that value, I get on my platform 18446744073709551615.












