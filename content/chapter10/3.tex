使用C++17中的执行策略,可以指定算法应该顺序运行、并行运行还是与向量化并行运行。

\begin{myTip}{并行STL的可用性}
2023年,只有微软编译器本地实现了STL算法的并行版本。对于GCC或Clang编译器,必须安装并使用\href{https://en.wikipedia.org/wiki/Threading_Building_Blocks}{Threading Building Blocks}。TBB是Intel为在多核处理器上并行编程而开发的C++模板库。
\end{myTip}

\mySamllsection{执行策略}

策略标签指定算法应该顺序运行、并行运行还是与矢量化并行运行。

\begin{itemize}
\item 
std::execution::seq: 串行运行算法

\item
std::execution::par: 多个线程上并行运行算法

\item
std::execution::par\_unseq: 多个线程上并行运行算法,并允许各个循环交错;允许有\href{https://en.wikipedia.org/wiki/SIMD}{SIMD}(单指令多数据)扩展的向量化版本。
\end{itemize}

下面的代码段显示了所有执行策略。

\filename{执行策略}

\begin{cpp}
std::vector<int> v = {1, 2, 3, 4, 5, 6, 7, 8, 9};

// standard sequential sort
std::sort(v.begin(), v.end());

// sequential execution
std::sort(std::execution::seq, v.begin(), v.end());

// permitting parallel execution
std::sort(std::execution::par, v.begin(), v.end());

// permitting parallel and vectorized execution
std::sort(std::execution::par_unseq, v.begin(), v.end());
\end{cpp}

该示例表明,可以在没有执行策略的情况下使用std::sort的重载。此外,在C++17中,可以显式指定应该使用顺序、并行还是并行和向量化版本。


\begin{myTip}{并行STL的可用性???}
	
算法是否以并行和向量化的方式运行取决于许多因素。例如,这取决于CPU和操作系统是否支持SIMD指令。此外,它还取决于编译器和编译代码的优化级别。

下面的示例展示了一个用于创建新vector的简单循环。

\begin{cpp}
const int SIZE= 8;

int vec[] = {1, 2, 3, 4, 5, 6, 7, 8};
int res[] = {0, 0, 0, 0, 0, 0, 0, 0};

int main(){
	for (int i = 0; i < SIZE; ++i) {
		res[i] = vec[i] + 5;
	}
}
\end{cpp}

表达式res[i] = vec[i] + 5是这个小示例中的关键行。有了\href{https://godbolt.org/}{Compiler Explorer},我们可以仔细查看由x86-64 clang 3.6生成的汇编指令。

\mySamllsection{无优化}

这是汇编说明,每个加法都是顺序完成的。

\myGraphic{0.6}{content/chapter10/images/2.jpg}{}


\mySamllsection{最大化优化}

使用最高的优化级别-O3,可以使用特殊的寄存器,例如可以保存128位或4 ‘int’的xmm0,在四个向量元素上并行地进行加法运算。

\myGraphic{0.6}{content/chapter10/images/3.jpg}{}

\end{myTip}

有77种STL算法可以通过执行策略进行参数化。

\mySamllsection{具有并行版本的算法}

下面是77种并行化的算法。

\begin{center}
77种并行化的算法
\end{center}

% Please add the following required packages to your document preamble:
% \usepackage{longtable}
% Note: It may be necessary to compile the document several times to get a multi-page table to line up properly
\begin{longtable}[c]{lll}
std::adjacent\_difference       & std::adjacent\_find             & std::all\_of                    \\
\endfirsthead
%
\endhead
%
std::any\_of                    & std::copy                       & std::copy\_if                   \\
std::copy\_n                    & std::count                      & std::count\_if                  \\
std::equal                      & std::exclusive\_scan            & std::fill                       \\
std::fill\_n                    & std::find                       & std::find\_end                  \\
std::find\_first\_of            & std::find\_if                   & std::find\_if\_not              \\
std::for\_each                  & std::for\_each\_n               & std::generate                   \\
std::inner\_product             & std::includes                   & std::inclusive\_scan            \\
std::is\_heap\_until            & std::inplace\_merge             & std::is\_heap                   \\
std::is\_sorted\_until          & std::is\_partitioned            & std::is\_sorted                 \\
std::merge                      & std::lexicographical\_compare   & std::max\_element               \\
std::mismatch                   & std::min\_element               & std::minmax\_element            \\
std::nth\_element               & std::move                       & std::none\_of                   \\
std::partition                  & std::partial\_sort              & std::partial\_sort\_copy        \\
std::remove                     & std::partition\_copy            & std::reduce                     \\
std::remove\_if                 & std::remove\_copy               & std::remov\_copy\_if            \\
std::replace\_copy\_if          & std::replace                    & std::replace\_copy              \\
std::replace\_copy              & std::replace\_if                & std::reverse                    \\
std::reverse\_copy              & std::rotate                     & std::rotate\_copy               \\
std::search                     & std::search\_n                  & std::set\_difference            \\
std::set\_intersection          & std::set\_symmetric\_difference & std::set\_union                 \\
std::sort                       & std::stable\_partition          & std::stable\_sort               \\
std::swap\_ranges               & std::transfrom                  & std::transform\_exclusive\_scan \\
std::transform\_inclusive\_scan & std::transform\_reduce          & std::uninitialized\_copy        \\
std::uninitialized\_copy\_n     & std::uninitialized\_fill        & std::uninitialized\_fill\_n     \\
std::unique                     & std::unique\_copy               &                                
\end{longtable}

\begin{myTip}{constexpr容器和算法}
C++20支持constexpr容器std::vector和std::string。constexpr表示,可以在编译时应用这两个容器的成员函数。此外,标准模板库的超过\href{https://en.cppreference.com/w/cpp/algorithm}{100个算法}可声明为constexpr。
\end{myTip}














