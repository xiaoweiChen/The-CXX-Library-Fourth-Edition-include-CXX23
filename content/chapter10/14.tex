数值算法std::accumulate、std::nearby\_difference、std::partial\_sum、std::inner\_product和std::iota,以及另外六个C++17算法std::exclusive\_scan、std::inclusive\_scan、std::transform\_exclusive\_scan、std::transform\_inclusive\_scan、std::reduce和std::transform\_reduce是特殊的,都在头文件<numeric>中定义。其广泛适用,可用可调用对象调用。

累加范围的元素。Init是起始值:

\begin{cpp}
T accumulate(InpIt first, InpIt last, T init)
T accumulate(InpIt first, InpIt last, T init, BiFun fun)
\end{cpp}

计算范围内相邻元素之间的差,并将结果存储在result中:

\begin{cpp}
OutIt adjacent_difference(InpIt first, InpIt last, OutIt result)
FwdIt2 adjacent_difference(ExePol pol, FwdIt first, FwdIt last, FwdIt2 result)

OutIt adjacent_difference(InpIt first, InpIt last, OutIt result, BiFun fun)
FwdIt2 adjacent_difference(ExePol pol, FwdIt first, FwdIt last,
						   FwdIt2 result, BiFun fun)
\end{cpp}

计算范围的部分和:

\begin{cpp}
OutIt partial_sum(InpIt first, InpIt last, OutIt result)
OutIt partial_sum(InpIt first, InpIt last, OutIt result, BiFun fun)
\end{cpp}

计算两个范围的内积(标量积)并返回结果:

\begin{cpp}
T inner_product(InpIt first1, InpIt last1, OutIt first2, T init)
T inner_product(InpIt first1, InpIt last1, OutIt first2, T init,
				BiFun fun1, BiFun fun2)
\end{cpp}

为范围中的每个元素赋一个依次递增的值。初始值为val:

\begin{cpp}
void iota(FwdIt first, FwdIt last, T val)
\end{cpp}

算法很难得到这样的结果。

不带可调用的std::accumulate可以使用以下策略:

\begin{cpp}
result = init;
result += *(first+0);
result += *(first+1);
...
\end{cpp}

不带可调用的std::adjacent\_difference可使用以下策略:

\begin{cpp}
*(result) = *first;
*(result+1) = *(first+1) - *(first);
*(result+2) = *(first+2) - *(first+1);
...
\end{cpp}

不带可调用的Std::partial\_sum可使用以下策略:

\begin{cpp}
*(result) = *first;
*(result+1) = *first + *(first+1);
*(result+2) = *first + *(first+1) + *(first+2)
...
\end{cpp}

具有挑战性的算法变化inner\_product(InpIt, InpIt, OutIt, T, BiFun fun1, BiFun fun2)具有两个二进制可调用对象,使用以下策略:第二个可调用对象fun2应用于每一对范围,以生成临时目标范围tmp,第一个可调用对象应用于目标范围tmp的每个元素,进行累积,从而生成最终结果。

\filename{数值算法}

\begin{cpp}
// numeric.cpp
...
#include <numeric>
...

std::array<int, 9> arr{1, 2, 3, 4, 5, 6, 7, 8, 9};
std::cout << std::accumulate(arr.begin(), arr.end(), 0); // 45
std::cout << std::accumulate(arr.begin(), arr.end(), 1,
							 [](int a, int b){ return a*b; } ); // 362880

std::vector<int> vec{1, 2, 3, 4, 5, 6, 7, 8, 9};
std::vector<int> myVec;
std::adjacent_difference(vec.begin(), vec.end(),
				std::back_inserter(myVec), [](int a, int b){ return a*b; });
for (auto v: myVec) std::cout << v << " "; // 1 2 6 12 20 30 42 56 72
std::cout << std::inner_product(vec.begin(), vec.end(), arr.begin(), 0); // 285

myVec= {};
std::partial_sum(vec.begin(), vec.end(), std::back_inserter(myVec));
for (auto v: myVec) std::cout << v << " "; // 1 3 6 10 15 21 28 36 45

std::vector<int> myLongVec(10);
std::iota(myLongVec.begin(), myLongVec.end(), 2000);
for (auto v: myLongVec) std::cout << v << " ";
						// 2000 2001 2002 2003 2004 2005 2006 2007 2008 2009
\end{cpp}

\mySamllsection{新并行算法与C++17}

通常用于并行执行的六种新算法称为\href{https://en.wikipedia.org/wiki/Prefix_sum}{前缀求和}。若给定的二元可调用对象不可关联和交换,则算法的行为未定义。

reduce: 减少范围内的元素。init是初始值。

\begin{itemize}
\item
其行为与std::accumulate相同,但范围可能会重新排列。
\end{itemize}

\begin{cpp}
ValType reduce(InpIt first, InpIt last)
ValType reduce(ExePol pol, InpIt first, InpIt last)

T reduce(InpIt first, InpIt last, T init)
T reduce(ExePol pol, InpIt first, InpIt last, T init)

T reduce(InpIt first, InpIt last, T init, BiFun fun)
T reduce(ExePol pol, InpIt first, InpIt last, T init, BiFun fun)
\end{cpp}

transform\_reduce: 转换和归约一个或两个范围的元素。Init是起始值。

\begin{itemize}
\item
行为类似于std::inner\_product,但范围可能会重新排列。

\item
若应用于两个范围
\begin{itemize}
\item
若没有提供,则使用乘法将范围转换为一个范围,使用加法将中间范围归约为结果

\item
若有提供,fun1用于变换步骤,fun2用于归约步骤
\end{itemize}

\item
若应用于单个范围
\begin{itemize}
\item
fun2用于转换给定的范围
\end{itemize}
\end{itemize}

\begin{cpp}
T transform_reduce(InpIt first, InpIt last, InpIt first2, T init)
T transform_reduce(InpIt first, InpIt last,
				   InpIt first2, T init, BiFun fun1, BiFun fun2)

T transform_reduce(FwdIt first, FwdIt last, FwdIt first2, T init)
T transform_reduce(ExePol pol, FwdIt first, FwdIt last,
				   FwdIt first2, T init, BiFun fun1, BiFun fun2)

T transform_reduce(InpIt first, InpIt last, T init, BiFun fun1, UnFun fun2)
T transform_reduce(ExePol pol, FwdIt first, FwdIt last,
				   T init, BiFun fun1, UnFun fun2)
\end{cpp}

\begin{myTip}{C++17中的MapReduce}

\href{https://www.haskell.org/}{Haskell}函数映射在C++中称为std::transform。当用std::transform\_reduce名称中的map替换transform时,将获得std::map\_reduce。\href{https://en.wikipedia.org/wiki/MapReduce}{MapReduce}是一个著名的并行框架,首先将每个值映射到一个新值,然后在第二阶段将所有值归约到结果。

该算法直接适用于C++17。在映射阶段,每个单词会映射到它的长度,然后在归约阶段,所有单词的长度会缩减为它们的和。结果是所有单词的长度之和。

\begin{cpp}
std::vector<std::string> str{"Only", "for", "testing", "purpose"};

std::size_t result= std::transform_reduce(std::execution::par,
				str.begin(), str.end(), 0,
				[](std::size_t a, std::size_t b){ return a + b; },
				[](std::string s){ return s.length(); });

std::cout << result << '\n'; // 21
\end{cpp}

\end{myTip}

exclusive\_scan: 使用二进制操作计算互斥前缀和。

\begin{itemize}
\item
行为类似于std::reduce,但提供了所有前缀和的范围

\item
排除每次迭代中的最后一个元素
\end{itemize}

\begin{cpp}
OutIt exclusive_scan(InpIt first, InpIt last, OutIt first, T init)
FwdIt2 exclusive_scan(ExePol pol, FwdIt first, FWdIt last,
					  FwdIt2 first2, T init)

OutIt exclusive_scan(InpIt first, InpIt last, OutIt first, T init, BiFun fun)
FwdIt2 exclusive_scan(ExePol pol, FwdIt first, FwdIt last,
					  FwdIt2 first2, T init, BiFun fun)
\end{cpp}

inclusive\_scan: 使用二元运算计算包含前缀和。

\begin{itemize}
\item
行为类似于std::reduce,但提供了所有前缀和的范围

\item
包括每次迭代中的最后一个元素
\end{itemize}

\begin{cpp}
OutIt inclusive_scan(InpIt first, InpIt last, OutIt first2)
FwdIt2 inclusive_scan(ExePol pol, FwdIt first, FwdIt last, FwdIt2 first2)

OutIt inclusive_scan(InpIt first, InpIt last, OutIt first, BiFun fun)
FwdIt2 inclusive_scan(ExePol pol, FwdIt first, FwdIt last,
					  FwdIt2 first2, BiFun fun)

OutIt inclusive_scan(InpIt first, InpIt last, OutIt firs2t, BiFun fun, T init)
FwdIt2 inclusive_scan(ExePol pol, FwdIt first, FwdIt last,
					  FwdIt2 first2, BiFun fun, T init)
\end{cpp}

transform\_exclusive\_scan: 首先变换每个元素,然后计算互斥前缀和。

\begin{cpp}
OutIt transform_exclusive_scan(InpIt first, InpIt last, OutIt first2, T init, BiFun fun, UnFun fun2)
FwdIt2 transform_exclusive_scan(ExePol pol, FwdIt first, FwdIt last, FwdIt2 first2, T init, BiFun fun, UnFun fun2)
\end{cpp}

transform\_inclusive\_scan: 首先变换输入范围中的每个元素,然后计算包含前缀和。

\begin{cpp}
OutIt transform_inclusive_scan(InpIt first, InpIt last, OutIt first2,
							   BiFun fun, UnFun fun2)

FwdIt2 transform_inclusive_scan(ExePol pol, FwdIt first, FwdIt last,
								FwdIt first2,
								BiFun fun, UnFun fun2)

OutIt transform_inclusive_scan(InpIt first, InpIt last, OutIt first2,
							   BiFun fun, UnFun fun2,
							   T init)

FwdIt2 transform_inclusive_scan(ExePol pol, FwdIt first, FwdIt last,
								FwdIt first2,
								BiFun fun, UnFun fun2,
								T init)
\end{cpp}

下面的示例说明了使用并行执行策略六种算法的用法。

\filename{新算法}

\begin{cpp}
// newAlgorithms.cpp
...
#include <execution>
#include <numeric>

...

std::vector<int> resVec{1, 2, 3, 4, 5, 6, 7, 8};
std::exclusive_scan(std::execution::par,
					resVec.begin(), resVec.end(), resVec.begin(), 1,
					[](int fir, int sec){ return fir * sec; });

for (auto v: resVec) std::cout << v << " "; // 1 1 2 6 24 120 720 5040

std::vector<int> resVec2{1, 2, 3, 4, 5, 6, 7, 8};

std::inclusive_scan(std::execution::par,
					resVec2.begin(), resVec2.end(), resVec2.begin(),
					[](int fir, int sec){ return fir * sec; }, 1);

for (auto v: resVec2) std::cout << v << " "; // 1 2 6 24 120 720 5040 40320

std::vector<int> resVec3{1, 2, 3, 4, 5, 6, 7, 8};
std::vector<int> resVec4(resVec3.size());
std::transform_exclusive_scan(std::execution::par,
							  resVec3.begin(), resVec3.end(),
							  resVec4.begin(), 0,
							  [](int fir, int sec){ return fir + sec; },
							  [](int arg){ return arg *= arg; });

for (auto v: resVec4) std::cout << v << " "; // 0 1 5 14 30 55 91 140

std::vector<std::string> strVec{"Only", "for", "testing", "purpose"};
std::vector<int> resVec5(strVec.size());
std::transform_inclusive_scan(std::execution::par,
							  strVec.begin(), strVec.end(),
							  resVec5.begin(),
							  [](auto fir, auto sec){ return fir + sec; },
							  [](auto s){ return s.length(); }, 0);

for (auto v: resVec5) std::cout << v << " "; // 4 7 14 21

std::vector<std::string> strVec2{"Only", "for", "testing", "purpose"};

std::string res = std::reduce(std::execution::par,
						strVec2.begin() + 1, strVec2.end(), strVec2[0],
						[](auto fir, auto sec){ return fir + ":" + sec; });

std::cout << res; // Only:for:testing:purpose

std::size_t res7 = std::transform_reduce(std::execution::par,
						strVec2.begin(), strVec2.end(), 0,
						[](std::size_t a, std::size_t b){ return a + b; },
						[](std::string s){ return s.length(); });

std::cout << res7; // 21
\end{cpp}














