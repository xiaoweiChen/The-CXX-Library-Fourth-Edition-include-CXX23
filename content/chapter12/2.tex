C++ inherited many numeric functions from C. They need the header \href{http://en.cppreference.com/w/cpp/numeric/math}{<cmath>}. The table below shows the names of these functions.

\begin{center}
Mathematical functions in <cmath>
\end{center}

% Please add the following required packages to your document preamble:
% \usepackage{longtable}
% Note: It may be necessary to compile the document several times to get a multi-page table to line up properly
\begin{longtable}[c]{lllll}
pow   & sin  & tanh  & asinh  & fabs  \\
\endfirsthead
%
\endhead
%
exp   & cos  & asin  & aconsh & fmod  \\
sqrt  & tan  & acos  & atanh  & frexp \\
log   & sinh & atan  & ceil   & ldexp \\
log10 & cosh & atan2 & floor  & modf 
\end{longtable}

Additionally, C++ inherits mathematical functions from C. They are defined in the header \href{http://en.cppreference.com/w/cpp/numeric/math}{<cstdlib>}.

Once more, here are the names.

\begin{center}
Mathematical functions in <cstdlib>
\end{center}

% Please add the following required packages to your document preamble:
% \usepackage{longtable}
% Note: It may be necessary to compile the document several times to get a multi-page table to line up properly
\begin{longtable}[c]{llll}
abs  & llabs & ldiv  & srand \\
\endfirsthead
%
\endhead
%
labs & div   & lldiv & rand 
\end{longtable}

All functions for integers are available for the types int, long, and ‘long long; all functions for floating-point numbers are available for the types float, double, and 'long double.

The numeric functions need to be qualified with the namespace std.

\filename{Mathematic functions}

\begin{cpp}
// mathFunctions.cpp
...
#include <cmath>
#include <cstdlib>
...

std::cout << std::pow(2, 10); // 1024
std::cout << std::pow(2, 0.5); // 1.41421
std::cout << std::exp(1); // 2.71828
std::cout << std::ceil(5.5); // 6
std::cout << std::floor(5.5); // 5
std::cout << std::fmod(5.5, 2); // 1.5

double intPart;
auto fracPart= std::modf(5.7, &intPart);
std::cout << intPart << " + " << fracPart; // 5 + 0.7
std::div_t divresult= std::div(14, 5);
std::cout << divresult.quot << " " << divresult.rem; // 2 4

// seed
std::srand(time(nullptr));
for (int i= 0;i < 10; ++i) std::cout << (rand()%6 + 1) << " ";
											// 3 6 5 3 6 5 6 3 1 5
\end{cpp}















